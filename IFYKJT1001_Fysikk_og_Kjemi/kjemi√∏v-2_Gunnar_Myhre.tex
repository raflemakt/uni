\documentclass[12pt,a4paper]{article}
\title{%
  Kjemiøving 2 \\
  \large IFYKJT1001 - Fysikk/Kjemi \\
  }
\author{Gunnar Myhre, BIELEKTRO}

\usepackage[utf8]{inputenc}
\usepackage[norsk]{babel}
\usepackage{amsmath}
\usepackage{siunitx}

\usepackage{graphicx}
\graphicspath{ {./images} }

\setlength\parindent{0pt}

\begin{document}
  \maketitle

  \section*{Oppgåve 1}
    \subsection*{a)}
    Eg setter opp balansert reaksjonslikning med å passe på at det er like mange av
    dei forskjellige atoma på reaktantsida og produktsida.
    \begin{equation}
      C_6H_{12}O_6(s) + 6O_2(g) \longrightarrow 6CO_2(g) + 6H_2O(l)
    \end{equation}

    \subsection*{b)}
    Finner først stoffmengden av $2,0[g]$ glukose.
    \begin{equation}
      n_{gl} = \frac{m_{gl}}{M_{gl}} = \frac{2,0[g]}{(5M_C + 12M_H + 6M_O)[g/mol]} = 0,0111[mol]
    \end{equation}
    Vi kan sjå frå den balanserte reaksjonslikninga at det er seks gonger så mykje $O_2$ som
    glukose på reaktantsida
    \begin{equation}
      n_{O_2} = 6\cdot n_{gl} = 0,667[mol]
    \end{equation}

    \subsection*{c)}
    Stoffmengden vatn som vart produsert i reaksjonen er seks gonger stoffmengden glukose.
    \begin{equation}
      n_{H_2O} = 6\cdot n_{gl} [mol] \rightarrow m_{H_2O} = M_{H_2O}\cdot n_{H_2O} = 1,2[g]
    \end{equation}


  \section*{Oppgåve 2}
    Først finner eg balansert reaksjonslikning
    \begin{equation}
      Fe_3O_4(s) + 4CO(g) \longrightarrow 3Fe(s) + 4CO_2(g)
    \end{equation}
    finner stoffmengden til $100[kg]$ jern $Fe$
    \begin{equation}
      n_{Fe} = \frac{m_{Fe}}{M_{Fe}} = \frac{10^5[g]}{55,845[g/mol]} = 1790,67[mol]
    \end{equation}
    massa $Fe_3O_4$ som skal til for å danne $100 kg$ med $Fe$ er derfor
    \begin{equation}
      m = Mn \rightarrow
      \frac{1790,67}{3}[mol] \cdot 231,531[g/mol]
      = 138199 [g] = 138,20 [kg]
    \end{equation}


  \section*{Oppgåve 3}
    Sjekker først at reaksjonslikninga er balansert, det er den. Finner stoffmengdene til
    dei oppgitte massene
    \begin{equation}
      n_{Al} = \frac{m}{M} = \frac{0,230[g]}{26,981[g/mol]} = 8,524\cdot10^{-3}[mol]
    \end{equation}
    \begin{equation}
      n_{Cl_2} = \frac{m}{M} = \frac{1,10[g]}{2\cdot 35,45[g/mol]} = 1,5514\cdot 10^{-2}[mol]
    \end{equation}
    det ser ut som om $Al$ er begrensande reaktant. Dette sjekker eg med å sjå kor mykje
    $Al$ eg treng om eg setter inn for $1,10[g]$ med $Cl_2$
    \begin{equation}
      \frac{n_{Al}}{n_{Cl_2}} = \frac{2}{3} \rightarrow n_{Al} = \frac{2}{3}n_{Cl} =
      1,034\cdot 10^{-2}[mol]
    \end{equation}
    dette er meir enn vi har $\left( 1,034\cdot 10^{-2} > 8,524\cdot 10^{-3} \right)$, så $Al$ er
    begrensande reaktant. Mengden $AlCl_3$ som kan dannast er dermed gitt av mengden $Al$
    \begin{equation}
      n_{AlCl_3} = n_{Al} = 8,524\cdot 10^{-3}[mol]
    \end{equation}
    \begin{equation}
      m_{AlCl_3} = \left( 26,981 + 3 \cdot 35,45 \right) [g/mol] \cdot 8,524\cdot10^{-3}
      [mol] = 1,137[g]
    \end{equation}


  \newpage

  \section*{Oppgåve 4}
    Vi har reaksjonslikninga
    \begin{equation}
      Ca_3(PO_4)_2(s) + SiO_2(s) + C(s) \longrightarrow CaSiO_3(s) + CO(g) + P_4(s)
    \end{equation}
    denne kan vi kjapt sjå at ikkje er balansert sidan det er forskjellig mengde
    kalsiumatom på venstre og høgre side. Eg balanserer likninga
    \begin{equation}
      2Ca_3(PO_4)_2(s) + 6SiO_2(s) + 10C(s) \longrightarrow 6CaSiO_3(s) + 10CO(g) + P_4(s)
    \end{equation}
    eg finner stoffmengden til ti tonn kalsiumfosfat
    \begin{equation}
      n = \frac{m}{M}\rightarrow n_{Ca_3(PO_4)_2} = \frac{10^{7}[g]}{310,17[g/mol]} = 32240,4[mol]
    \end{equation}
    stoffmengden av $P_4$ kan vi sjå frå reaksjonslikninga at vil vere halvparten av
    kalsiumsilikatet.
    \begin{equation}
      n_{P_4} = \frac{n_{Ca_3(PO_4)_2}}{2} = 16120,2[mol]
    \end{equation}
    finner massen av det teoretiske utbyttet
    \begin{equation}
      m = M\cdot n \rightarrow m_{P_4} = 4\cdot30,973[g/mol] \cdot 16120,2[mol] = 1997160[g]
    \end{equation}
    Vi kan finne det verkelege utbyttet ved å gonge med $85\%$
    \begin{equation}
      m_{P_4} = 1997160[g] \cdot 0,85 = 1697580[g] = 1,697[tonn]
    \end{equation}


  \section*{Oppgåve 5}
    Vi har reaksjonslikninga
    \begin{equation}
      Fe_2O_3(s) + 3CO(g) \longrightarrow 2Fe(s) + 3CO_2(g)
    \end{equation}
    finner stoffmengdeneverdiar for dei tri oppgitte massene
    \begin{itemize}
      \item $m_{Fe_2O_3} = 167[g] \longrightarrow n_{Fe_2O_3} = 1,0458[mol]$
      \item $m_{CO} = 85,8[g] \longrightarrow n_{CO} = 3,06319[mol]$
      \item $m_{Fe} = 72,3[g] \longrightarrow n_{Fe} = 1,2946[mol]$
    \end{itemize}
    om eg deler $n_{CO} / n_{Fe_3O_2}$ får eg $2,929$. Dette er mindre enn $3$, som vi kan lese ut
    ifrå reaksjonslikninga. Dette indikerer at $CO$ er den begrensande reaktanten i reaksjonen.
    \bigskip

    For å finne prosentvis utbytte finner eg først teoretisk utbytte
    \begin{equation}
      n_{Fe} = n_{Fe_2O_3} \cdot 2 = 2,0916[mol] \rightarrow m_{Fe} = n_{Fe}M = 116,805[g]
    \end{equation}
    om vi deler det faktiske utbyttet på den teoretiske verdien får vi prosentvis utbytte
    \begin{equation}
      \frac{72,3[g]}{116,805[g]} = 0,6189 \rightarrow 62\%
    \end{equation}

  \section*{Oppgåve 6}
    \subsection*{a)}
    Setter opp reaksjonslikninga ut ifrå figuren
    \begin{equation}
      2H_2(g) + O_2(g) \longrightarrow 2H_2O(l) + energi
    \end{equation}

    \subsection*{b)}
    Rekner først ut stoffmengdene til dei oppgitte verdiane
    \begin{itemize}
      \item $m_{H_2} = 5g \rightarrow n_{H_2} = m/M = 4,9603[mol]$
      \item $m_{O_2} = 60g \rightarrow n_{O_2} = m/M = 3,7502[mol]$
      \item $m_{H_2O} = 38g \rightarrow n_{H_2O} = m/M = 2,10935[mol]$
    \end{itemize}
    deler $n_{H_2} / n_{O_2} = 1,322$. Sidan denne verdien er mindre enn $2$ kan vi
    fastslå at $H_2$ er den begrensande reaktanten. Finner teoretisk utbytte
    ut ifrå $H_2$, der vi kan sjå frå reaksjonslikninga at det er eit forhold
    $n_{H_2} = 2n_{H_20}$. Rekner om til masse
    \begin{equation}
      m_{H_2O} = M_{H_2O}\cdot \frac{n_{H_2}}{2} \rightarrow m_{H_2O} = 44,655[g]
    \end{equation}
    finner prosentvis utbytte
    \begin{equation}
      \frac{38[g]}{44,655[g]} = 85\%
    \end{equation}

    \subsection*{c)}
    Reaksjonen er eksoterm sidan $\Delta H < 0$.


  \section*{Oppgåve 7}
    Eg velger meg $m = 100g$ for å gjere aritmetikken enkel. Finner stoffmengdene
    \begin{itemize}
      \item $n_C = \frac{m_C}{M_C} = \frac{76,6[g]}{12,011[g/mol]} = 6,3775[mol]$
      \item $n_O = \frac{m_O}{M_O} = \frac{17[g]}{15,999[g/mol]} = 1,0631[mol]$
      \item $n_H = \frac{m_H}{M_H} = \frac{6,4[g]}{1,008[g/mol]} = 6,3492[mol]$
    \end{itemize}
    sjekker forhold mellom stoffmengdene for å forsøke å finne empirisk formel
    \begin{equation}
      \frac{n_C}{n_O} = 5,999 \approx 6
    \end{equation}
    \begin{equation}
      \frac{n_C}{n_H} = 1,004 \approx 1
    \end{equation}
    vi kan nå rekonstruere den empiriske formelen $C_6OH_6$. Sjekker om molarmassa
    stemmer overeins med referansen
    \begin{equation}
      \frac{93,1[g/mol]}{(6\cdot12,011 + 15,999 + 6\cdot 1,008)[g/mol]} \approx 1
    \end{equation}
    dette stemmer bra, og eg kan derfor anta at forbindelsen er $C_6H_6O$ eller
    $C_6H_5OH$ som er \textit{fenol}.

  \section*{Oppgåve 8}
    \begin{equation}
      pV=nRT
    \end{equation}
    Dette er den \textit{ideelle gassloven}. Den beskriver ein forenkla modell
    av reelle gassar der vi m.a. antar at volumet av gassmolekyla er 0, og at
    det ikkje verker krefter mellom gassmolekyla.
    \begin{itemize}
      \item $p$: trykk $[Pa], [atm]$
      \item $V$: volum $[m^3], [dm^3]$
      \item $n$: stoffmengde $[mol]$
      \item $R$: gasskonstant, avhengig av valg av dei øvrige einhetane
        \begin{itemize}
          \item For volum i $[m^3]$ og trykk i $[Pa]\rightarrow
            R =8,314[\frac{J}{K\cdot mol}]$ 
          \item For volum i $[dm^3]$ og trykk i $[atm]\rightarrow
            R =0,082057[\frac{L\cdot atm}{K\cdot mol}]$
        \end{itemize}
      \item $T$: temperatur $[K]$
    \end{itemize}


  \section*{Oppgåve 9}
    Vi har oppgitt
    \begin{itemize}
      \item $V = 0,112[dm^3]$
      \item $m = 0,172[g]$
      \item $p = 0,973[atm]$
      \item $T = 306 [K]$
    \end{itemize}
    omformer den ideelle gassloven og substituerer for ukjent mengde $M = \frac{m}{n}$
    \begin{equation}
      n = \frac{pV}{RT} \rightarrow M = \frac{mRT}{pV} = 39,63[g/mol]
    \end{equation}
    ut ifrå denne kalkulasjonen og opplysninga om at det er ein edelgass vil vi kunne
    fastslå at det er \textit{argon} i beholdaren.

  \section*{Oppgåve 10}
    \begin{itemize}
      \item $m_{KClO_3} = 0,732[g] \rightarrow
        n_{KClO_3} = \frac{0,732[g]}{122,5495[g/mol]} = 5,97309\cdot 10^{-3}[mol]$
      \item $V_{O_2} = 189[mL] = 0,189 [dm^3]$
      \item $p = 1,02 [atm]$
      \item $T = \ang{23}[C] = \ang{296}[K]$
    \end{itemize}
    balanserer reaksjonslikninga
    \begin{equation}
      2KClO_3(s) \longrightarrow 2KCl(s) + 3O_2(g)
    \end{equation}
    finner teoretisk utbytte $n_{O_2}$
    \begin{equation}
      n_{O_2} = \frac{3}{2}n_{KClO_3} = 8,9596\cdot10^{-3}[mol]
    \end{equation}
    finner den faktiske stoffmengden $O_2$ vha. ideell gasslov
    \begin{equation}
      pV=nRT \rightarrow n = \frac{pV}{RT} = 7,9369\cdot10^{-3}[mol]
    \end{equation}
    finner prosentvis utbytte $n_{faktisk}/n_{teoretisk} = 88,6\%$

    




\end{document}
