\documentclass[12pt,a4paper]{article}
\title{%
  Øving 3 \\
  \large IFYKJT1001 - Fysikk/Kjemi \\
  }
\author{Gunnar Myhre, BIELEKTRO}

\usepackage[utf8]{inputenc}
\usepackage[norsk]{babel}
\usepackage{amsmath}
\usepackage{siunitx}

\usepackage{graphicx}
\graphicspath{ {./images} }

\setlength\parindent{0pt}

\begin{document}
  \maketitle

  \section*{Oppgåve 1}
    \subsection*{a)}
    Krafta i positiv retning  er gitt ved Hooks Lov:
    \begin{equation}
      F = kx \rightarrow F(x) = (200N/m)x
    \end{equation}
    Derom vi antar at fjøra ikkje yter noko kraft på isblokka etter $0,025m$ kan vi skrive
    summen av det kinetiske arbeidet som
    \begin{equation}
      W = \int_{0m}^{0,025m}(200N/m)xdx \rightarrow \frac{200N/m}{3200N/m} = 0,063 J
    \end{equation}

    \subsection*{b)}
    Setter opp energiloven for systemet
    \begin{equation}
      \frac{1}{2}mv_1^2 +
      \frac{1}{2}kx_1^2 =
      \frac{1}{2}mv_0^2 + 
      \frac{1}{2}kx_0^2
    \end{equation}
    setter $x_0 = -0,025$. Fjerner ledd som er null, $v_0 = 0$ og $x_1 = 0$
    \begin{equation}
      \frac{1}{2}v_1^2m = kx_0^2 \rightarrow v_1 = \sqrt{\frac{200}{6400}} = \frac{\sqrt{2}}{8}
    \end{equation}
    farta når isblokka slipper fjøra er $0,18 m/s$

  \section*{Oppgåve 2}
    Arbeid-/energisetninga seier
    \begin{equation}
      W = \Delta E_K \rightarrow
      W =
      \frac{1}{2}mv_1^2 - 
      \frac{1}{2}mv_0^2
    \end{equation}
    vi veit også at 
    \begin{equation}
     W = \int_0^{8,6m}F(x)dx \rightarrow
      \left[ 6,0\frac{1}{3}x^3 - 2\frac{1}{2}x^2 + 6x \right] _0^{8,6} = 1249,75J
    \end{equation}
    desse to likningene kan vi slå saman og finne $v_1$
    \begin{equation}
      \frac{1}{2}mv_1^2 - \frac{1}{2}mv_0^2 = 1249,75J
    \end{equation}
    forenkler algebraisk og setter inn kjente verdiar
    \begin{equation}
      v = \sqrt{\frac{2\cdot 1249,75}{5,5}} \rightarrow  v = 21 m/s
    \end{equation}


  \section*{Oppgåve 3}
    Setter opp energiloven for systemet, der indeks $1$ er for likevektspunktet til fjøra
    og $0$ er for posisjonen når fjøra er samanklemt
    \begin{equation}
      \frac{1}{2}mv_1^2 + 
      \frac{1}{2}ky_1^2 -
      mgy_1 =
      \frac{1}{2}mv_0^2 +
      \frac{1}{2}ky_0^2 -
      mgy_0
    \end{equation}
    fjerner ledda som inneholder $v_0 = 0$ og $y_1 = 0$
    \begin{equation}
      \frac{1}{2}mv_1^2 =
      \frac{1}{2}ky_0^2 -
      mgy_0
    \end{equation}
    vi kjenner alle mengdene untatt $v_1$, som er farta når mursteinen forlater
    likevektspunktet til fjøra. Denne kan vi finne vha. den tidlause bevegelseslikninga
    for konstant akselerasjon
    \begin{equation}
      v_{topp}^2 - v_1^2 = 2as \rightarrow v_1 = \sqrt{2mg}
    \end{equation}
    vi står igjen med polynomet
    \begin{equation}
      \frac{1}{2}ky_0^2 - mgy_0 - m^2g = 0
    \end{equation}
    løyser og får $y_0 = 0,57 \vee y_0 = -0,49$. Det negative svaret er ugyldig, og
    derfor er $y_0 = 0,57m$.


  \section*{Oppgåve 4}
    \subsection*{a)}
    Eg dekomponerer $\vec{P}$ og lister resten av informasjonen i oppgåveteksten
    \begin{itemize}
      \item $P_x = Pcos(\ang{30})$
      \item $P_y = Psin(\ang{30})$
      \item $m=20,0kg$
      \item $s=8,0m$
      \item $v_0=0,459m/s$
      \item $v_1=1,92m/s$
    \end{itemize}
    finner akselerasjonen vha. tidlaus bevegelseslikning for konstant akselerasjon
    \begin{equation}
      v^2 - v_0^2 = 2as \rightarrow a = \frac{v_1^2 - v_0^2}{2s} = 0,21723 m/s^2
    \end{equation}
    finner friksjonskrafta vha. Newtons andre lov langs x-aksa
    \begin{equation}
      \Sigma F = ma \rightarrow Pcos(\ang{30}) - f = ma
      \rightarrow f = Pcos(\ang{30}) - ma = 125,559N
    \end{equation}
    arbeidet utført er gitt ved
    \begin{equation}
      W=Fs \rightarrow W=125,559N \cdot (-8,0m) = -1004J
    \end{equation}

    \subsection*{b)}
    Normalkrafta er er i dette tilfellet motkraft til gravitasjonskrafta $mg$ og
    den eksterne krafta i y-retning, $P_y$
    \begin{equation}
      N = mg + Psin(\ang{30}) = 271,2N
    \end{equation}
    vi finner friksjonskoeffisienten ved å sjå på friksjonskrafta under kinetisk friksjon
    som er gitt ved
    \begin{equation}
      f=\mu N \rightarrow \mu = \frac{f}{N} = 0,46
    \end{equation}

    \subsection*{c)}
    Effekten utøvd av friksjonskrafta er gitt som
    \begin{equation}
      P=Fv \rightarrow P = 125,6N \cdot \frac{1,92m/s + 0,459m/s}{2} = 149,4 W
    \end{equation}


  \section*{Oppgåve 5}
    Vi kan bruke formel for arbeid frå gravitasjonen
    \begin{equation}
      W = -mg(y_B - y_A) \rightarrow W = 170mg
    \end{equation}
    med $100\%$ effektivitet vil massen som trengs vere
    \begin{equation}
      m = \frac{W}{170g} = 1,199 \cdot 10^3 m^3/s
    \end{equation}
    ved $92\%$ effektivitet vil massen som trengs vere
    \begin{equation}
      \frac{1,199 \cdot 10^3 m^3/s}{0,92} = 1,304 \cdot 10^3m^3/s
    \end{equation}


  \section*{Oppgåve 6}
    Arbeidet tyngdekrafta utfører på kula er
    \begin{equation}
      W = -mgy_1 + mgy_0
    \end{equation}
    der det andre leddet faller bort sidan $y_0 = 0$. Det mekaniske arbeidet på kanonkula
    er gitt ved
    \begin{equation}
      W = \frac{1}{2}v_1^2 - \frac{1}{2}v_0^2
    \end{equation}
    Frå dette får vi eit uttrykk for $y_1$, som er toppen av kulebanen
    \begin{equation}
      y_1 = \frac{v_0^2}{2g} - \frac{v_1^2}{2g}
    \end{equation}
    Vi er kun ute etter bevegelsen i y-retning sidan det kun er tyngdekrafta som verker
    på kanonkula. Derfor kan vi substituere $v_{0_y} = vsin(\ang{60})$ og $v_{1_y} = 0$
    \begin{equation}
      y_1 = \frac{(15sin(\ang{60})m/s)^2}{2g} = 8,6m
    \end{equation}


  \section*{Oppgåve 7}
    Vi kan teste dette numerisk vha. formelen frå førrige oppgåve
    \begin{itemize}
      \item $y_A = \frac{(15sin(\ang{90})m/s)^2}{2g} = 11,5m$
      \item $y_B = \frac{(15sin(\ang{60})m/s)^2}{2g} = 8,6m$
      \item $y_C = \frac{(15sin(\ang{30})m/s)^2}{2g} = 2,9m$
      \item $y_D = \frac{(15sin(\ang{15})m/s)^2}{2g} = 0,77m$
    \end{itemize}
    og sette inn i bevegelseslikninga $t=\frac{2s}{v+v_0}$
    \begin{itemize}
      \item $t_A = \frac{2\cdot s}{v+v_0} = \frac{23,0m}{225m/s} = 0,10s$
      \item $t_B = \frac{2\cdot s}{v+v_0} = \frac{17,2m}{169/s} = 0,10s$
      \item $t_C = \frac{2\cdot s}{v+v_0} = \frac{5,80m}{56,3m/s} = 0,10s$
      \item $t_D = \frac{2\cdot s}{v+v_0} = \frac{1,54m}{15,1/s} = 0,10s$
    \end{itemize}
    Kulene treffer bakken samtidig


  \section*{Oppgåve 8}
    \subsection*{a)}
    Energiloven for denne fjøra gjev oss
    \begin{equation}
      \frac{1}{2}mv_1^2 +
      \frac{1}{2}kx_1^2 =
      \frac{1}{2}mv_0^2 +
      \frac{1}{2}kx_0^2 
    \end{equation}
    setter inn for kjente mengder og får
    \begin{equation}
      v_1 = \sqrt{\frac{kx_1^2}{m}} = 3,11
    \end{equation}

    \subsection*{b)}
    Krafta langs skråplanet finner vi ved å dekomponere gravitasjonskrafta
    \begin{equation}
      F = -mgsin(\ang{37})
    \end{equation}
    vi finner akselerasjonen
    \begin{equation}
      F=ma \rightarrow a = \frac{-mgsin(\ang{37})}{2,00kg} = -5,903 m/s^2
    \end{equation}
    bruker den tidlause bevegelseslikninga for konstant akselerasjon
    \begin{equation}
      s = \frac{v^2 - v_0^2}{2a} \rightarrow s = 0,82m
    \end{equation}
    
    \subsection*{c)}
    Sidan systemet er modellert med veldig få krefter (bl.a. utan friksjon og luftmotstand)
    vil all energien forbli i boksen. Den vil omgjere all energien frå potensiell til
    kinetisk og attende, og fortsette slik for alltid.
\end{document}
