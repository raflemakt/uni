\documentclass[12pt,a4paper]{article}
\title{%
  Øving 7 \\
  \large IELET1001 - Elektroteknikk \\
  }
\author{Gunnar Myhre, BIELEKTRO}

\usepackage[utf8]{inputenc}
\usepackage[norsk]{babel}
\usepackage[siunitx]{circuitikz}
\usepackage{amsmath}
\usepackage{pgfplots}

\usepackage{graphicx}
\graphicspath{ {./images} }

\setlength\parindent{0pt}

\begin{document}
  \maketitle

  \section*{Oppgåve 1}
  \subsection*{a)}
    Eg tolker oppgåveteksten til at nodespenninga $v(t)$ er gitt i noda som
    binder RLC-elementene saman (merka med $2$ på teikninga).
    \bigskip

    Når brytaren lukkast ved $t=0$ står vi igjen med ein parallell RLC.
    Ved $t = 0^-$ vil spenninga over kondensatoren vere $4V$.
    Sidan spenninga over kondensatoren er kontinuerleg vil
    \begin{equation}
      v(0^+) = v(0^-) = 4V
    \end{equation}
    Ved $t=0^+$ vil dei energilagrande elementene oppføre seg som kjelder.

    \begin{center}
      \begin{circuitikz}[american] \draw
        (0,0) to[I, l=0<\ampere>, i_<=$i_L$] (0,3) -- (2,3)
              to[R, l=0.5<\ohm>, i=$i_R$, v=$v_R$] (2,0) -- (0,0)
        (2,3) -- (4,3)
              to[V, l=4<\volt>, i>=$i_C$] (4,0) -- (2,0)
        ;
      \end{circuitikz}
    \end{center}
    frå straum-/spenningsforholdet for kondensator kjenner vi at
    \begin{equation}
      i_c=C\frac{dv_c}{dt} \rightarrow \frac{dv_c}{dt} = \frac{i_c}{C}
    \end{equation}
    KCL i den øverste noda gjev oss
    \begin{equation}
      i_L + i_R + i_C = 0 \rightarrow
      0A +\frac{4V}{0,5\si{\ohm}} + C\frac{dv_c}{dt} = 0
    \end{equation}
    som gjev oss
    \begin{equation}
      \frac{dv(0^+)}{dt} = -8V/s
    \end{equation}

  \subsection*{b)}
    Finner uttrykk for straumane
    \begin{itemize}
      \item $i_R = \frac{v(t)}{R}$
      \item $i_L = i_L(0) + \frac{1}{L}\int_0^t{v(t)dt}$
      \item $i_C = C\frac{dv_c}{dt}$
    \end{itemize}
    setter inn i KCL
    \begin{equation}
      \frac{v(t)}{R} + \frac{1}{L}\int_0^t{v(t)dt} + C\frac{dv(t)}{dt} = 0
    \end{equation}
    deriverer med hensyn på $t$
    \begin{equation}
      C\frac{d^2v(t)}{dt^2} +
      \frac{1}{R}\frac{dv(t)}{dt} +
      \frac{1}{L}v(t)
    \end{equation}
    får høgaste deriverte aleine
    \begin{equation}
      \frac{d^2v(t)}{dt^2} +
      \frac{1}{RC}\frac{dv(t)}{dt} +
      \frac{1}{LC}v(t) = 0
    \end{equation}
    setter inn for $R=0,5$, $L=0,25$ og $C=1$ og finner karakteristisk likning
    \begin{equation}
      v'' + 2v' + 4v = 0 \rightarrow r^2 + 2r + 4 = 0
    \end{equation}
    finner røttene til karakteristisk likning
    \begin{equation}
      r_{1,2} = -1 \pm \sqrt{3}j
    \end{equation}
    Sidan røttene til den karakteristiske likninga er eit komplekskonjugert par vil
    transienten til spenninga $v(t)$ vere av underdempa karakter. Frå matematikken
    kjenner vi den generelle løysninga på ei slik differensiallikning som
    \begin{equation}
      r_{1,2} = \alpha \pm i\beta \rightarrow
      y(x) = e^{\alpha x} \left[ A_1 cos(\beta x) + A_2 sin(\beta x) \right]
    \end{equation}
    For å finne koeffisientane $A_1$ og $A_2$ finner vi først $v(t)$ og $v'(t)$
    \begin{itemize}
      \item $v(t) = A_1 cos(\sqrt{3}t) e^{-t} + A_2 sin(\sqrt{3}t) e^{-t}$
      \item $v'(t) = -A_1 e^{-t} \left[ cos(\sqrt{3}t)+sin(\sqrt{3}t)\sqrt{3} \right] +
        A_2 e^{-t} \left[ cos(\sqrt{3}t)\sqrt{3} - sin(\sqrt{3}t) \right]$
    \end{itemize}
    setter inn for initialbetingelsane $v(0) = 4V$ og $v'(0) = -8V/s$
    \begin{itemize}
      \item $4 = A_1$
      \item $-8 = -A_1 + \sqrt{3}A_2$
    \end{itemize}
    Løyser likningssettet og får $A_1 = 4$, $A_2 = -\frac{4\sqrt{3}}{3} \approx -2,31$
    \begin{equation}
      v(t) = 4cos(\sqrt{3}t)e^{-t} - 2,31sin(\sqrt{3}t)e^{-t}
    \end{equation}



  \section*{Oppgåve 2}
    Denne oppgåva er også ein parallell RLC-krets og vi kan derfor bruke delar av
    utrekninga frå oppgåve 1. Likninga er på forma
    \begin{equation}
      \frac{d^2v(t)}{dt^2} +
      \frac{1}{RC}\frac{dv(t)}{dt} +
      \frac{1}{LC}v(t) = 0
    \end{equation}
    setter inn for $R=5$, $L=2$ og $C=\frac{1}{40}$ og finner karakteristisk likning
    \begin{equation}
      v'' + 8v' + 20v = 0 \rightarrow
      r^2 + 8r + 20 = 0
    \end{equation}
    vi finner røttene til den karakteristiske likninga vha. andregradsformelen
    og finner den generelle differensiallikninga for $v(t)$
    \begin{equation}
      r_{1,2} = -4 \pm 2j
      \rightarrow v(t) = A_1 cos(2t)e^{-4t} + A_2 sin(2t)e^{-4t}
    \end{equation}
    vi deriverer $v(t)$
    \begin{equation}
      v'(t) = -A_1 e^{-4t} \left[ 4cos(2t)+sin(2t)2 \right] +
        A_2 e^{-4t} \left[ cos(2t)2 - 4sin(2t) \right]
    \end{equation}
    koeffisientane $A_1$ og $A_2$ kan vi utlede frå initialbetingelsane. Vi kjenner
    $v(0) = 10V$ og $\frac{dv(0)}{dt}$ kan vi finne frå straum-/spenningsforholdet
    til kondensatoren
    \begin{equation}
      KCL: i_L + i_R + i_C = 0  \rightarrow 1A + \frac{10V}{5\si{\ohm}} = -i_C
      \rightarrow i_C = -3A
    \end{equation}
    \begin{equation}
      i_C = C\frac{dv(t)}{dt} \rightarrow
      \frac{dv(0)}{dt} = \frac{-3A}{1/40F} = -120V/s
    \end{equation}
    setter inn for initialbetingelsane i $v(t)$ og $v'(t)$
    \begin{itemize}
      \item $A_1 = 10$
      \item $2A_2 - 4A_1 = -120$
    \end{itemize}
    løyser og finner $A_1 = 10$, $A_2 = -40$. Setter inn og finner den partikulære
    løysninga for $v(t)$
    \begin{equation}
      v(t) = 10 cos(2t)e^{-4t} - 40 sin(2t)e^{-4t}
    \end{equation}

  \newpage


  \section*{Oppgåve 3}
    Dette er kretsen for $t \ge 0$
    \begin{center}
      \begin{circuitikz}[american] \draw
        (0,0) to[V, l=27<\volt>, i>=$i(t)$, invert] (0,4)
              to[R, l=6<\ohm>] (2,4)
              to[R, l=1.5<\ohm>] (4,4)
              to[C, l=0.02<\farad>, v=$v_c$] (4,0) -- (2,0)
              to[L, l=0.25<\henry>, v=$v_l$] (0,0)
        ;
      \end{circuitikz}
    \end{center}
    setter opp KVL i maska
    \begin{equation}
      v_r + v_c + v_l = 27V
    \end{equation}
    finner uttrykk for spenningene
    \begin{itemize}
      \item $v_r = 7,5i(t)$
      \item $v_l = L\frac{di(t)}{dt}$
      \item $v_c = v_c(0^-) + \frac{1}{C}\int_0^t{i(t)dt}$
    \end{itemize}
    setter inn i KVL
    \begin{equation}
      7,5i(t) + L\frac{di(t)}{dt} + v_c(0^-) + \frac{1}{C}\int_0^t{i(t)dt} = 27V
    \end{equation}
    setter høgaste deriverte aleine, deriverer med hensyn på t
    \begin{equation}
      i'' + \frac{R}{L}i' + \frac{1}{LC}i = 0
    \end{equation}
    fyller inn for $R = 7,5$, $L = 0,25$ og $C = 0,02$ og finner karakteristisk
    likning
    \begin{equation}
      r^2 + 30r + 200 = 0
    \end{equation}
    finner røttene $r = -20$ og $r = -10$. Dette gjev oss den generelle
    differensiallikninga
    \begin{equation}
      i(t) = K_1 e^{-20t} + K_2 e^{-10t}
    \end{equation}
    deriverer $i(t)$
    \begin{equation}
      i'(t) = -20K_1 e^{-20t} - 10K_2 e^{-10t}
    \end{equation}
    vi finner initialbetingelsane for $t=0^+$. På tidspunktet $t=0^-$ er kretsen
    stasjonær. Kondensatoren er ein åpen krets og spolen er ein kortslutning.
    Derfor er straumen i heile kretsen gitt ved
    \begin{equation}
      i(0^-) = \frac{v}{R} = \frac{33}{6} = 5,5
    \end{equation}
    sidan straumen i spolen er kontinuerleg er også $i(0^+) = 5,5$. KVL i kretsen
    ved $t=0^+$ er gitt ved
    \begin{equation}
      -27 + 7,5\cdot5,5 - 6 + v_l = 0
    \end{equation}
    ved straum-/spenningsforholdet til spole får vi at
    \begin{equation}
      \frac{di(0^+)}{dt} = \frac{v_l(0^+)}{L} = -33
    \end{equation}
    setter inn i $i(t)$ og $i'(t)$ for å finne $A_1$ og $A_2$
    \begin{itemize}
      \item $i(0^+) = 5,5 = K_1 + K_2$
      \item $i'(0^+) = -33 = -20K_1 + -10K_2$
    \end{itemize}
    løyser likningssettet og finner $K_1 = -2,2$, $K_2 = 7,7$. Dette kan vi sette
    inn i $i(t)$ og finner den partikulære løysninga
    \begin{equation}
      i(t) = -2,2e^{-20t} + 7,7 e^{-10t}
    \end{equation}

  \newpage

  \section*{Oppgåve 4}
    Om vi forenkler kretsen til éi maske og finner eit uttrykk for $i(t)$ i denne 
    eine maska kan vi til slutt finne eit uttrykk for $v_0(t)$ over motstanden vha.
    ohms lov. Med andre ord er vi ikkje opptatt av anna enn éin motstand, éin spole
    og éin kondensator i serie med ein energikjelde.
    \bigskip

    For å skrive kretsen vår om på denne måten bruker eg kjeldetransformasjon av
    straumkjelda ein rekke gonger, og står til slutt igjen med ein spenningskjelde
    $v_s = 4V$ og ein motstand $r = 4 \si{\ohm}$ i serie. Denne motstanden slår eg
    saman med motstanden $r_0$ på høgre sida. For $t \ge 0$ ser kretsen slik ut
    \begin{center}
      \begin{circuitikz}[american] \draw
        (0,0) to[V, l=4<\volt>, invert] (0,3)
              to[L, l=0.05<\henry>] (3,3)
              to[C, l=0.008<\farad>] (6,3)
              to[R, l=5<\ohm>] (6,0) -- (0,0)
        ;
      \end{circuitikz}
    \end{center}
    KVL i maska gjev oss
    \begin{equation}
      v_R + v_L + v_C = 4V
    \end{equation}
    desse spenningene kan vi beskrive ved straum-/spenningsforhold
    \begin{itemize}
      \item $v_R = Ri(t)$
      \item $v_L = L\frac{di(t)}{dt}$
      \item $v_C = v_C(0) + \frac{1}{C}\int_0^t{i(t)dt}$
    \end{itemize}
    vi veit at spenninga over kondensatoren i $t = 0^-$ er $0V$. Setter
    opp generell differensiallikning
    \begin{equation}
      L\frac{di(t)}{dt} + Ri(t) + \frac{1}{C}\int_0^t{i(t)dt} = 4
    \end{equation}
    setter høgaste ordens deriverte aleine og deriverer dt
    \begin{equation}
      i'' + \frac{R}{L}i' + \frac{1}{LC}i = 0
    \end{equation}
    om vi løyser den karakteristiske likninga ser vi at vi får ei dobbelrot
    \begin{equation}
      r^2 + 100r + 2500 = 0 \rightarrow r = -50
    \end{equation}
    det er dermed snakk om ein transient av kritisk dempa karakter. Den 
    generelle differensiallikninga for dette kjenner vi som
    \begin{equation}
      i(t) = B_1 e^{-50t} + B_2 t e^{-50t}
    \end{equation}
    deriverer $i(t)$
    \begin{equation}
      i'(t) = -50B_1 e^{-50t} + B_2te^{-50t} - 50B_2e^{-50t}
    \end{equation}
    for å finne koeffisientane må vi finne initialbetingelsane $i(0^+)$ og
    $i'(0^+)$. Pga. at straum i spolar er kontinuerleg vil
    \begin{equation}
      i(0^+) = i(0^-) = 0
    \end{equation}
    for å finne $i'(0^+)$ må vi finne $v_l(0^+)$ i den originale teikninga. Sidan
    spenninga i kondensatoren er kontinuerleg vil den fortsatt vere $0V$ i $t=0^+$.
    \begin{center}
      \begin{circuitikz}[american] \draw
        (0,0) to[R, l=4<\ohm>, *-*] (3,0)
              to[L, l=0.025<\henry>, v=$v_l$] (6,0)
              to[C, l=0.008<\farad>, *-*] (9,0)

        (0,0) node[label={[font=\footnotesize]100:$4V$}] {}
        (3,0) node[label={[font=\footnotesize]100:$4V$}] {}
        (6,0) node[label={[font=\footnotesize]100:$0V$}] {}
        (10,0) node[label={[font=\footnotesize]100:$0V$}] {}
        ;
      \end{circuitikz}
    \end{center}
    straumen $i(0^+)$ er som kjent $0$ så det er ingen spenningsfall over motstanden,
    dermed er $v_l = 4V$. Nå kan vi finne $i'(0^+)$
    \begin{equation}
      \frac{di(0^+)}{dt} = \frac{v_l}{L} = \frac{4}{1/20} = 80
    \end{equation}
    nå som vi kjenner to initialbetingelsar kan vi sette inn i $i(t)$ og $i'(t)$
    \begin{itemize}
      \item $B_1 = 0$
      \item $-50B_1 + B_2 = 80$
    \end{itemize}
    den partikulære løysninga av $i(t)$ er derfor
    \begin{equation}
      i(t) = 80 t e^{-50t}
    \end{equation}
    vi finner spenninga $v_0(t)$ vha. ohms lov
    \begin{equation}
      v(t) = R\cdot i(t) \rightarrow v_0(t) = 80 t e^{-50t}
    \end{equation}

    


  \section*{Oppgåve 5}
    I oppgåve 1 utleda eg formelen for ein parallell RLC-krets
    \begin{equation}
      \frac{d^2v(t)}{dt^2} +
      \frac{1}{RC}\frac{dv(t)}{dt} +
      \frac{1}{LC}v(t) = 0
    \end{equation}
    Eg velger meg $R = 220\si{\kilo\ohm}$
    \begin{itemize}
      \item $4\cdot 10^7 = \frac{1}{220 \cdot 10^3 C}
        \Rightarrow C = 88 \cdot 10^{-9}$
      \item $3\cdot 10^{14} = \frac{1}{88\cdot10^{-9}L}
        \Rightarrow L = 264 \cdot 10^{-3}$
    \end{itemize}
    denne kretsen oppfyller kriteriene
    \begin{center}
      \begin{circuitikz}[american] \draw
        (0,0) to[R, l=220<\kilo\ohm>] (0,2) -- (3,2)
              to[L, l=264<\milli\henry>] (3,0) -- (0,0)
        (3,2) -- (6,2)
              to[C, l=88<\nano\farad>] (6,0) -- (3,0)
        ;
      \end{circuitikz}
    \end{center}


  \section*{Oppgåve 6}
    \subsection*{a)}
    Sidan L og C står i parallell stemmer dette forholdet:
    \begin{equation}
      v_c = v_L = L\frac{di_L}{dt}
    \end{equation}
    KVL i kretsen for $t > 0$ gjev oss
    \begin{equation}
      V_s = v_c + R(i_L + i_c)
    \end{equation}
    $i_l$ og $i_c$ er gitt ved straum-/spenningsforhold for spole og kondensator
    \begin{itemize}
      \item $i_c = C\frac{dv_c}{dt}$
      \item $i_L = \frac{1}{C}\int_0^t v_c(t) dt$
    \end{itemize}
    i eitt uttrykk vert dette
    \begin{equation}
      RC\frac{dv_c}{dt} + v_c + \frac{R}{L} \int_0^t v_c(t) dt = V_s
    \end{equation}
    setter høgaste ordens deriverte for seg sjølv og deriverer dt
    \begin{equation}
      v_c'' + \frac{1}{RC}v_c' + \frac{1}{LC}v_c = 0
    \end{equation}

    \subsection*{b)}
    Vi kan ta utgongspunkt i denne formelen frå oppgåve \textbf{a}
    \begin{equation}
      RC\frac{dv_c}{dt} + v_c + \frac{R}{L} \int_0^t v_c(t) dt = V_s
    \end{equation}
    målet er å lage ein formel som er avhengig av $i(t)$, så vi substituerer
    $v_c = L\frac{di_L}{dt}$
    \begin{equation}
      LRC\frac{d^2i_L}{dt^2} + L\frac{di_L}{dt} + \frac{R}{L}\int_0^t L\frac{di_L}{dt}dt = V_s
    \end{equation}
    integralet av ein konstant forenklast til
    \begin{equation}
      LRC\frac{d^2i_L}{dt^2} + L\frac{di_L}{dt} + \frac{R}{L}Li_L = V_s
    \end{equation}
    setter høgaste ordens deriverte aleine og forenkler leibnizformalismen
    \begin{equation}
      i_L'' + \frac{1}{RC}i_L' + \frac{1}{LC}i_L = \frac{V_s}{RLC}
    \end{equation}

    eg merker meg at koeffisientane til differensiallikninga er like,
    uavhengig av om vi går ut ifrå $i_L$ eller $v_c$
    \begin{itemize}
      \item $i_L'' + \frac{1}{RC}i_L' + \frac{1}{LC}i_L = \frac{V_s}{RLC}$
      \item $v_c'' + \frac{1}{RC}v_c' + \frac{1}{LC}v_c = 0$
    \end{itemize}






\end{document}
