\documentclass[12pt,a4paper]{article}
\title{%
  Øving 3 \\
  \large IELET1001 - Elektroteknikk \\
  }
\author{Gunnar Myhre, BIELEKTRO}

\usepackage[utf8]{inputenc}
\usepackage[norsk]{babel}
\usepackage[siunitx]{circuitikz}
\usepackage{amsmath}

\setlength\parindent{0pt}

\begin{document}
  \maketitle
    
  \section{Oppgåve 1}
    Bruker maskestraum
    \begin{center}
      \begin{circuitikz}[american] \draw 
        (0,0) to[I, l=1<\ampere>] (0,3) -- (4,3)
              to[R, l=3<\ohm>, v=$v_2$, *-*] (8,3) -- (12,3)
              to[I, l=3<\ampere>] (12,0) --(8,0) -- (4,0) -- (0,0)
        (8,0) to[R, l=1<\ohm>, v<=$v_3$] (8,3)
        (4,0) to[R, l=2<\ohm>, v<=$v_1$] (4,3)
        ;
        \draw[->,shift={(2,1.5)}] (120:.7cm) arc (120:-90:.7cm) node at(0,0){$I_a$};
        \draw[->,shift={(6,1.5)}] (120:.7cm) arc (120:-90:.7cm) node at(0,0){$I_b$};
        \draw[->,shift={(10,1.5)}] (120:.7cm) arc (120:-90:.7cm) node at(0,0){$I_c$};
      \end{circuitikz}
    \end{center}
    \begin{itemize}
      \item $I_a = 1\si{\ampere}$
      \item $KVL_b: 6\si{\ohm}I_b - 2\si{\ohm}I_a - 1\si{\ohm}I_c=0$
      \item $I_c = 3\si{\ampere}$
    \end{itemize}
    Løyser likningssettet og finner at $I_b = 5/6A = 0,833A$. Vi finner spenningsfalla
    \begin{equation}
      V_1 = 2\si{\ohm}(I_a-I_b)=1/3\si{\volt}
    \end{equation}
    \begin{equation}
      V_2 = 3\si{\ohm}(5/6)=5/2\si{\volt}
    \end{equation}
    \begin{equation}
      V_3 = 1\si{\ohm}(I_b-I_c)=-13/6\si{\volt}
    \end{equation}
    Dette kan vi sjekke med KVL i maske b:
    \begin{equation}
      -V_1 + V_2 + V_3 = 0 \rightarrow -1/3 + 5/2 - 13/6 = 0
    \end{equation}
    $V_1 = 1/3V = 0,333V$

  \section{Oppgåve 2}
    \begin{center}
      \begin{circuitikz}[american] \draw 
        (0,0) to[V, l=45<\volt>, invert] (0,3)
              to[R, l=3<\kilo\ohm>, v=$v_1$, *-*] (4,3)
              to[R, l=2<\kilo\ohm>, v=$v_2$, *-*] (8,3)
              to[short, *-o] (10,3)
        (8,3) to[I, l=6<\milli\ampere>] (8,0)
              to[short, *-o] (10,0)
        (8,0) -- (4,0) -- (0,0)
        (4,0) to[R, l=6<\kilo\ohm>] (4,3)
        (10,3)to[open, v=$v_0$] (10,0)
              
        ;
        \draw[->,shift={(2,1.5)}] (120:.7cm) arc (120:-90:.7cm) node at(0,0){$I_a$};
        \draw[->,shift={(6,1.5)}] (120:.7cm) arc (120:-90:.7cm) node at(0,0){$I_b$};
      \end{circuitikz}
    \end{center}
    \begin{itemize}
      \item $KVL_a: -45 + 9kI_a - 6kI_b = 0$
      \item $I_b = 6mA$
    \end{itemize}
    Løyser og får $I_a = 9mA$. Finner spenningsfalla frå $45V$-kilden bort til $v_0^+$
    \begin{equation}
      v_1 = 3\cdot9\si{\volt} = 27\si{\volt}
    \end{equation}
    \begin{equation}
      v_2 = 2\cdot6\si{\volt} = 12\si{\volt}
    \end{equation}
    \begin{equation}
      v_0^+ = 45\si{\volt} - 27\si{\volt} - 12\si{\volt} = 6\si{\volt}
    \end{equation}
    $v_0 = 6\si{\volt}$

  \newpage

  \section{Oppgåve 3}
    \begin{center}
      \begin{circuitikz}[american] \draw 
        (0,0) to[I, l=45<\milli\ampere>] (0,3) -- (3,3) -- (3,5)
              to[I, l=5<\milli\ampere>] (7,5) -- (7,3)
              to[R, l=2<\kilo\ohm>, *-*] (11, 3)
              to[short, *-o] (13,3)
        (3,3) to[R, l=6<\kilo\ohm>, *-*] (7,3)
              to[R, l=6<\kilo\ohm>, *-*] (7,0) -- (11,0)
              to[short, *-o] (13,0)
        (11,0) -- (7,0) -- (0,0) node[ground]{} (0,-2)
        (3,3) to[R, l=3<\kilo\ohm>, *-*] (3,0)
        (11,3) to[R, l=3<\kilo\ohm>, *-*] (11,0)

        (3,3) node[label={[font=\footnotesize]150:$v_1$}] {}
        (7,3) node[label={[font=\footnotesize]150:$v_2$}] {}
        (11,3) node[label={[font=\footnotesize]150:$v_0$}] {}
        ;
      \end{circuitikz}
    \end{center}
    \begin{center}
      \begin{itemize}
        \item $KCL_1: -45mA + 5mA + \frac{v_1}{3\si{\kilo\ohm}} + \frac{v_1-v_2}{3\si{\kilo\ohm}}=0
          \rightarrow 3v_1-v_2=240V$
        \item $-5mA+\frac{v_2-v_1}{6\si{\kilo\ohm}}+\frac{v_2}{6\si{\kilo\ohm}}+
          \frac{v_2-v_3}{2\si{\kilo\ohm}}=0 \rightarrow -v_1+5v_2-3v_3=30V$
        \item $\frac{v_3-v_2}{2\si{\kilo\ohm}}+\frac{v_3}{1\si{\kilo\ohm}}=0
          \rightarrow -v_2+3v_3=0$
      \end{itemize}
    \end{center}
    løyser likningssettet
    \begin{equation}
      \begin{bmatrix}
        3  & -1 &  0 \\
        -1 &  5 & -3 \\
        0  & -1 &  3
      \end{bmatrix}
      \begin{bmatrix}
        v_1 \\
        v_2 \\
        v_3
      \end{bmatrix}
      =
      \begin{bmatrix}
        240 \\
        30 \\
        0
      \end{bmatrix}
    \end{equation}
    $v_1 = 90V$, $v_2 = 30V$, $v_3 = 10V$

  \section{Oppgåve 4}
    Ved KCL i noda mellom straumkildene kan vi sjå at straumen igjennom $V_0$ er $2mA$.
    Sidan straumen entrer på negativ side får vi at $V_0 = -2mA\cdot4\si{\kilo\ohm}=-8V$.

  \section{Oppgåve 5}
    Setter opp maskestraum, merker at maske a og c må skrivast som supermaske
    \begin{center}
      \begin{circuitikz}[american] \draw 
        (0,0) to[V, l=12<\volt>, invert] (0,3) -- (0,6)
              to[R, l=1<\kilo\ohm>] (8,6) -- (8,3)
              to[R, l=1<\kilo\ohm>, v=$V_0$] (8,0) -- (0,0)
        (4,0) to[R, l=1<\kilo\ohm>, *-*] (4,3)
        (0,3) to[R, l=1<\kilo\ohm>, *-*, i=$I_x$] (4,3)
              to[cI, l=$4I_x$, *-*] (8,3)
        ;
        \draw[->,shift={(4,4.5)}] (120:.7cm) arc (120:-90:.7cm) node at(0,0){$I_a$};
        \draw[->,shift={(2,1.5)}] (120:.7cm) arc (120:-90:.7cm) node at(0,0){$I_b$};
        \draw[->,shift={(6,1.5)}] (120:.7cm) arc (120:-90:.7cm) node at(0,0){$I_c$};
      \end{circuitikz}
    \end{center}
    \begin{itemize}
      \item $KVL_b: 2I_b -I_a -I_c = 12$
      \item $KVL_{a+c}: 2I_a -2I_b + 2I_c = 0$
      \item $implisitt: 4I_x = I_c - I_a$
      \item $implisitt: I_x = I_b - I_a$
    \end{itemize}
    Løyser likningssettet og finner $I_c = -6mA$. Då er $V_o=1\si{\kilo\ohm}\cdot(-6)mA=-6V$
    
  \newpage

  \section{Oppgåve 6}
    Setter opp maskestraum

    \begin{center}
      \begin{circuitikz}[american] \draw 
        (0,3) to[cI, l=$2I_x$](0,0)
              to[R, l=1<\kilo\ohm>] (0,-3) -- (4,-3) -- (8,-3)
              to[R, l=1<\kilo\ohm>] (8,0)
              to[R, l=1<\kilo\ohm>, i<=$I_x$] (4,0)
              to[V, l=13<\volt>, invert] (4,-3)
        (0,0) to[R, l=1<\kilo\ohm>] (4,0)
              to[R, l=1<\kilo\ohm>] (4,3)
        (0,3) -- (4,3) -- (8,3)
              to[I, l=9<\milli\ampere>] (8,0)

        ;
        \draw[->,shift={(2,1.5)}] (120:.7cm) arc (120:-90:.7cm) node at(0,0){$I_a$};
        \draw[->,shift={(6,1.5)}] (120:.7cm) arc (120:-90:.7cm) node at(0,0){$I_b$};
        \draw[->,shift={(2,-1.5)}] (120:.7cm) arc (120:-90:.7cm) node at(0,0){$I_d$};
        \draw[->,shift={(6,-1.5)}] (120:.7cm) arc (120:-90:.7cm) node at(0,0){$I_c$};
      \end{circuitikz}
    \end{center}
    \begin{equation}
      implisitt_a: I_a = -2I_x
    \end{equation}
    \begin{equation}
      implisitt_b: I_b = 9\si{\milli\ampere}
    \end{equation}
    \begin{equation}
      KVL_c: 13\si{\milli\ampere} +2I_c -1I_b = 0
    \end{equation}
    \begin{equation}
      KVL_d: -13\si{\milli\ampere} +2I_d -2I_a = 0
    \end{equation}
    \begin{equation}
      implisitt: I_x = I_c - I_b
    \end{equation}
    straumen igjennom motstanden som har spenningsfallet $V_0$ er $I_c$. Løyser likningssettet
    og finner at $I_c = -2\si{\milli\ampere}$. Med ohms lov finner vi at
    \begin{equation}
      V_0 = 1\si{\kilo\ohm}\cdot(-2\si{\milli\ampere}) = -2\si{\volt}
    \end{equation}

  \section{Oppgåve 7}
    Summen av spenningene i den ytterste sløyfa må vere null, uavhengig av resten av kretsen.
    Derfor er $-10V+2V+V_0=0\rightarrow V_0=8V$

  \section{Oppgåve 8}
    Velger jord i den midtre noda og markerer dei tri andre vesentlege nodene.
    \begin{center}
      \begin{circuitikz}[american] \draw 
        (0,0) to[R, l=2<\kilo\ohm>] (0,3)
              to[R, l=1<\kilo\ohm>] (0,6) -- (6,6)
              to[V, l=6<\volt>, i_>=$I_o$] (6,0)
        (3,6) to[R, l=2<\kilo\ohm>] (3,3)
              to[R, l=1<\kilo\ohm>] (3,0)
        (3,3) to[V, l=10<\volt>, invert, *-*] (0,3)
        (0,0) -- (6,0)
        (2.5,3) -- node[ground]{} (2.5,2)

        (0,3) node[label={[font=\footnotesize]150:$V_1$}] {}
        (3,6) node[label={[font=\footnotesize]150:$V_2$}] {}
        (3,0) node[label={[font=\footnotesize]150:$V_3$}] {}
        ;
      \end{circuitikz}
    \end{center}
    Setter opp KCL supernode 2+3.
    \begin{equation}
      \frac{V_2-V_1}{1\si{\kilo\ohm}}+\frac{V_2}{2\si{\kilo\ohm}}+I
      -I+\frac{V_3}{1\si{\kilo\ohm}}+\frac{V_3-V_1}{2\si{\kilo\ohm}}=0
    \end{equation}
    forenkler til
    \begin{equation}
      -3V_1+3V_2+3V_3=0
    \end{equation}
    Det er implisitt i teikninga at
    \begin{equation}
      V_2 = V_3 + 6V
    \end{equation}
    og at
    \begin{equation}
      V_1 = 10V
    \end{equation}
    Finner $V_3 = 2V$ ved å løyse likningssettet. Dermed kan vi finne $I_o$
    ved å skrive KCL i node 3
    \begin{equation}
      -I_o + \frac{2V}{1\si{\kilo\ohm}}+\frac{2V-10V}{2\si{\kilo\ohm}}=0
      \rightarrow I_o = 2mA-4mA =-2mA
    \end{equation}
    $I_0$ er $-2mA$

  \section{Oppgåve 9}
    Løyser vha. maskestraum.
    \begin{center}
      \begin{circuitikz}[american] \draw 
        (0,0) to[V, l=10<\volt>, invert] (0,3)
              to[R, l=5<\ohm>] (4,3)
              to[R, l=1<\ohm>] (8,3)
              to[R, l=3<\ohm>] (12,3)
              to[R, l=4<\ohm>] (12,0) --(8,0) -- (4,0) -- (0,0)
        (8,0) to[R, l=2<\ohm>] (8,3)
        (4,0) to[R, l=2<\ohm>] (4,3)
        ;
        \draw[->,shift={(2,1.5)}] (120:.7cm) arc (120:-90:.7cm) node at(0,0){$I_a$};
        \draw[->,shift={(6,1.5)}] (120:.7cm) arc (120:-90:.7cm) node at(0,0){$I_b$};
        \draw[->,shift={(10,1.5)}] (120:.7cm) arc (120:-90:.7cm) node at(0,0){$I_c$};
      \end{circuitikz}
    \end{center}
    \begin{itemize}
      \item $KVL_a: -10 + 7I_a -2I_b = 0$
      \item $KVL_b: 5I_b-2I_a-2I_c=0$
      \item $KVL_c: 9I_c-2I_b=0$
    \end{itemize}
    Løyser likningssettet og finner $I_a = 1,6334A$ og $I_c=0,1593A$.
    \begin{itemize}
      \item $I_y=I_c=1,1593A$
      \item $P=vi\rightarrow P=10V\cdot1,6334A=16,33W$
    \end{itemize}

  \section{Oppgåve 10}
    \begin{center}
      \begin{circuitikz}[american] \draw 
        (0,0) to[V, l_=212<\volt>] (4,0)
              to[R, l=5<\ohm>, i<=$i_{5\si{\ohm}}$] (4,3)
              to[R, l=2<\ohm>, i<=$i_{2\si{\ohm}}$] (4,6) -- (0,6) -- (0,3) -- (0,0)
        (0,3) to[R, l=3<\ohm>] (4,3)
              to[R, l=3<\ohm>] (8,3)
              to[V, l=122<\volt>] (8,0) -- (4,0)
        ;
        \draw[->,shift={(6,1.5)}] (120:.7cm) arc (120:-90:.7cm) node at(0,0){$I_3$};
        \draw[->,shift={(2,1.5)}] (120:.7cm) arc (120:-90:.7cm) node at(0,0){$I_2$};
        \draw[->,shift={(2,4.5)}] (120:.7cm) arc (120:-90:.7cm) node at(0,0){$I_1$};
      \end{circuitikz}
    \end{center}
    \begin{itemize}
      \item $5I_1-3I_2=0$
      \item $8I_2-3I_1-5I_3 = 212$
      \item $8I_3-5I_2=-122$
    \end{itemize}
    Løyser likningssettet og finner $I_1 =26,49A$, $I_2 = 44,15A$ og $I_3 = 12,34A$.
    \begin{itemize}
      \item $I_{2\si{\ohm}}=I_1= 26,49A$
      \item $I_{2\si{\ohm}}=I_2-I_3= 44,15A-12,34A = 31,81A$
    \end{itemize}

  \section{Oppgåve 11}
    Vi ser av KCL at $I_x$ også må gå mot venstre over $2\si{\kilo\ohm}$-motstanden.
    For å bruke maskestraum i denne oppgåva må vi bruke supermaske. Den ekvivalente
    kretsen vert då ein enkeltmaskekrets.
    \begin{equation}
      -2I_x +4I_x -12+2I_x = 0 \rightarrow I_x = 3mA
    \end{equation}
    spenninga $V_0$ over motstanden vert då $v=Ri\rightarrow
    v=2mA\cdot 3\si{\kilo\ohm} = 6V$

  \section{Oppgåve 12}
    Setter opp maskestraum
    \begin{center}
      \begin{circuitikz}[american] \draw 
        (0,0) to[V, l=6<\volt>, i>=$I_o$, invert] (0,3) -- (0,6)
              to[cI, l=$2I_x$] (8,6) -- (8,3)
              to[cV, l=$2V_x$] (8,0) -- (0,0)
        (0,3) to[R, l=1<\kilo\ohm>, v=$V_x$] (4,3)
              to[R, l=1<\kilo\ohm>] (8,3)
        (4,0) to[R, l=1<\kilo\ohm>, i<=$I_x$] (4,3)
        ;
        \draw[->,shift={(4,4.5)}] (120:.7cm) arc (120:-90:.7cm) node at(0,0){$I_a$};
        \draw[->,shift={(2,1.5)}] (120:.7cm) arc (120:-90:.7cm) node at(0,0){$I_b$};
        \draw[->,shift={(6,1.5)}] (120:.7cm) arc (120:-90:.7cm) node at(0,0){$I_c$};
      \end{circuitikz}
    \end{center}
    \begin{itemize}
      \item $KVL_b: 2I_b-I_a-I_c=6mA$
      \item $KVL_c: 2V_x +2I_c -I_b -I_a = 0$
      \item $implisitt: I_x = I_b - I_c$
      \item $implisitt: I_a = 2I_x$
      \item $implisitt: V_x=I_b-I_a$
    \end{itemize}
    Løyser likningssettet og finner at $I_b = I_o = 9,6mA$

\end{document}
