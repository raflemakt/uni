\documentclass[12pt,a4paper]{article}
\title{%
  Øving 8 \\
  \large IELET1001 - Elektroteknikk \\
  }
\author{Gunnar Myhre, BIELEKTRO}

\usepackage[utf8]{inputenc}
\usepackage[norsk]{babel}
\usepackage[siunitx]{circuitikz}
\usepackage{amsmath}

\usepackage{graphicx}
\graphicspath{ {./images} }

\setlength\parindent{0pt}

\begin{document}
  \maketitle

  \section*{Oppgåve 1}
    Gitt den periodiske funksjonen
    \begin{equation}
      v(t) = 25 cos(30t + \ang{10})V
    \end{equation}
    er frekvensen gitt som 
    \begin{equation}
      f = \frac{\omega}{2\pi} = 15\pi
    \end{equation}
    og perioden
    \begin{equation}
      T = \frac{1}{f} = \frac{1}{15\pi}
    \end{equation}


  \section*{Oppgåve 2}
    Fasevinklane mellom dei to spenningene er 
    \begin{equation}
      \ang{40} - (-\ang{20}) = \ang{60}
    \end{equation}

  \section*{Oppgåve 3}
    Vi veit generelt at om
    \begin{equation}
      v(t) = V_m cos(\omega t + \phi) = Re \{V_m e^{j\phi}e^{j\omega t} \}
    \end{equation}
    så er fasevektoren til $v(t)$
    \begin{equation}
      V=V_m e^{j\phi}
    \end{equation}
    \begin{itemize}
      \item \textbf{a)} $v(t) = 20 cos(2t-\ang{10}) \rightarrow 20e^{\ang{-10}j}$
      \item \textbf{b)} $i(t) = -4sin(10t+\ang{35}) = 4cos(10t+\ang{125}) \rightarrow
        4e^{\ang{125}j}$
      \item \textbf{c)} $v(t) = 60 sin(5t - \ang{25}) = 60cos(5t - \ang{115}) \rightarrow
        60e^{\ang{-115}j}$
    \end{itemize}


  \section*{Oppgåve 4}
    \begin{equation}
      Z = \left( ( (( (2j-j) || (2||2)) + 18) || 2j ) + 2 \right) \si{\ohm}
    \end{equation}
    \begin{equation}
      Z = \left( ( (( j || 1)) + 18) || 2j ) + 2 \right) \si{\ohm}
    \end{equation}
    \begin{equation}
      Z = \left( ( \frac{j}{1+j} + 18) || 2j ) + 2 \right) \si{\ohm}
    \end{equation}
    \begin{equation}
      Z = \left( \frac{(\frac{j}{1+j} + 18)2j}{2j + \frac{j}{1+j} + 18}+ 2 \right) \si{\ohm}
    \end{equation}
    gonger med konjugat for å fjerne brøk
    \begin{equation}
      Z = \left( \frac{(\frac{1}{2}j + \frac{1}{2} + 18)2j}
      {2j + \frac{1}{2}j + \frac{1}{2} + 18}+ 2 \right) \si{\ohm}
    \end{equation}
    \begin{equation}
      Z = \left( \frac{j^2 + 37j}{\frac{5}{2}j + \frac{37}{2}}+ 2 \right) \si{\ohm}
    \end{equation}
    gonger med konjugat for å fjerne brøk
    \begin{equation}
      Z = \left( \frac{-1 + 37j}{\frac{5}{2}j + \frac{37}{2}}
      \frac{(\frac{5}{2}j - \frac{37}{2})}{(\frac{5}{2}j - \frac{37}{2})}
      + 2 \right) \si{\ohm}
    \end{equation}
    \begin{equation}
      Z = \left( \frac{74 + 687j}{348,5} + 2 \right) \si{\ohm}
    \end{equation}
    \begin{equation}
      Z = \left( 2,21 + 1,97j \right) \si{\ohm}
    \end{equation}


  \newpage


  \section*{Oppgåve 5}
    \begin{center}
      \begin{circuitikz}[american] \draw
        (0,0) to[V, l=$\textbf{V}$, i_>=$I$, invert] (0,3)
              to[R, l=5<\ohm>] (3,3)
              to[L, l=$Z_L$] (3,0)
              to[C, l=$Z_C$] (0,0)
        ;
      \end{circuitikz}
    \end{center}
    Vi kjenner til desse verdiane
    \begin{itemize}
      \item $\textbf{V} = 13e^{j\ang{85}} = 13cos\ang{85} + j13sin\ang{85} = 1,133 + j12,95$
      \item $\omega = 600$
      \item $Z_R = 5$
      \item $Z_L = j\omega L = j600L$
      \item $Z_C = \frac{1}{j\omega C} = -20,08j$
    \end{itemize}
    At fasevinkelen mellom $V$ og $I$ er lik er eit spesialtilfelle som \textit{kun} inntrer 
    om den totale reaktansen i kretsen er null, med andre ord når $Z_L + Z_C = 0$. 
    \begin{equation}
      \frac{1}{j\omega C} + j\omega L = 0 \rightarrow
      \frac{1}{\omega C} = -j^2\omega L \rightarrow
      L = \frac{1}{C\omega^2}
    \end{equation}
    Sidan $C = 83\si{\micro\farad}$ er gitt må vi velge $L = 33,47 \si{\milli\henry}$.

    \newpage

  \section*{Oppgåve 6}
    \begin{center}
      \begin{circuitikz}[american] \draw 
        (0,0) to[V, l=$V$, invert] (0,3)
              to[R, l=$R_1$, *-*] (4,3)
              to[R, l=$R_2$, *-*] (8,3)
              to[C, l=$C$, v=$V_C$] (8,0)
        (8,0) -- (4,0) -- (0,0)
        (4,0) to[L, l=$L$, v<=$V_L$] (4,3)
              
        ;
        \draw[->,shift={(2,1.5)}] (120:.7cm) arc (120:-90:.7cm) node at(0,0){$I_1$};
        \draw[->,shift={(6,1.5)}] (120:.7cm) arc (120:-90:.7cm) node at(0,0){$I_2$};
      \end{circuitikz}
    \end{center}

    Løyser vha. maskestraum. KVL i maske 1 og 2 gjev oss matrisa
    \begin{equation}
      A\vec{x} = \vec{b} \rightarrow
      \left[ \begin{array}{cc}
        R_1 + j\omega L, & -j\omega L \\
        - j\omega L, & R_2 + \frac{1}{j\omega C} + j\omega L \\
      \end{array} \right]
      \cdot
      \left[ \begin{array}{c}
        I_1 \\
        I_2 \\
      \end{array} \right]
      =
      \left[ \begin{array}{c}
        24\angle{\ang{60}} \\
        0 \\
      \end{array} \right]
    \end{equation}
    fører opp kjente verdiar for fasevektorane
    \begin{itemize}
      \item $V = 24\angle \ang{60} = 24cos(\ang{60}) + j24sin(\ang{60}) = 12 + j12\sqrt{3}$
      \item $Z_{R_1} = 4$
      \item $Z_{R_2} = 8$
      \item $Z_L = j\omega L = j6$
      \item $Z_C = \frac{1}{j\omega C} = -j4$
    \end{itemize}
    fører inn i matrisa
    \begin{equation}
      A\vec{x} = \vec{b} \rightarrow
      \left[ \begin{array}{cc}
        4 + j6, & -j6 \\
        -j6, & 8 + j2 \\
      \end{array} \right]
      \cdot
      \left[ \begin{array}{c}
        I_1 \\
        I_2 \\
      \end{array} \right]
      =
      \left[ \begin{array}{c}
        12 + j12\sqrt{3} \\
        0 \\
      \end{array} \right]
    \end{equation}
    Sidan dette er ei $2\times2$-matrise kan vi løyse den vha. Cramers regel. Då må vi først
    finne determinanten til A
    \begin{equation}
      det(A) = (4+j6)(8+j2)-(-j6)(-j6) = 56+56j
    \end{equation}
    finner $I_1$ vha. Cramers regel
    \begin{equation}
      I_1 = \frac{
        \left| \begin{array}{cc}
          12+j12\sqrt{3}, & -j6 \\
          0,              & 8+j2 \\
        \end{array} \right|}
      {det(A)} =
      \frac{(12+j12\sqrt{3})(8+j2)}{56+56j} \approx 2,18 + j1,21
    \end{equation}
    finner $I_2$ vha. Cramers regel
    \begin{equation}
      I_2 = \frac{
        \left| \begin{array}{cc}
          4+j6, & 12+j12\sqrt{3} \\
          -j6,  & 0 \\
        \end{array} \right|}
      {det(A)} =
      \frac{0-(-j6)(12+j12\sqrt{3})}{56+56j} \approx -0,47 + j1,75
    \end{equation}
    for å vinne spenningene $V_L$ og $V_C$ bruker vi den utvida Ohms' lov $V=ZI$
    \begin{equation}
      V_C = -j4 \cdot I_2 = -j4(-0,47 + j1,75) = 7 + j1,88
    \end{equation}
    \begin{equation}
      V_C = j6 \cdot (I_1 - I_2) = j6(2,18+j1,21 - (-0,47 +j1,75)) = 3,24 + j15,9
    \end{equation}
    tilbakeført til tidsdomenet vert dette
    \begin{itemize}
      \item $I_1 = 2,18+j1,21 \longrightarrow i_1(t) = 2,49cos(\omega t + \ang{61}) [A]$
      \item $I_2 = -0,47+j1,75 \longrightarrow i_2(t) = 1,81cos(\omega t + \ang{165}) [A]$
      \item $V_L = 3,24+j15,9 \longrightarrow v_L(t) = 16,22cos(\omega t + \ang{12}) [V]$
      \item $V_C = 7+j1,88 \longrightarrow v_C(t) = 7,25cos(\omega t + \ang{75}) [V]$
    \end{itemize}


  \section*{Oppgåve 7}
    Velger meg jord i den nederste noda. Vi får kun éi vesentlig node (supernode) som
    består av $V_O$ og noda heilt til venstre i underkant av $6V$-spenningskjelda
    \begin{equation}
      \frac{V_O - 6}{2} + \frac{(V_O - 6) - 6V_O}{2} + \frac{V_O - 6V_O}{j} + \frac{V_O}{2} = 0
    \end{equation}
    forenkler algebraisk
    \begin{equation}
      V_O = \frac{-36 - j120}{109}[V]
    \end{equation}

  \newpage

  \section*{Oppgåve 8}
    Løyser vha. maskestraum
    \begin{itemize}
      \item $KVL_{I_x}: 2I_x -jI_x +4jI_x -I_1 = 0$
      \item $KVL_{I_1}: 2I_1 -I_x -4I_x = 6$
    \end{itemize}
    Løyser som matrise $A\vec{x} = \vec{b}$
    \begin{equation}
      A\vec{x} = \vec{b} \rightarrow
      \left[ \begin{array}{cc}
        -1, & 2+3j \\
        2,  & -5 \\
      \end{array} \right]
      \cdot
      \left[ \begin{array}{c}
        I_x \\
        I_1 \\
      \end{array} \right]
      =
      \left[ \begin{array}{c}
        0 \\
        6 \\
      \end{array} \right]
    \end{equation}
    finner determinanten til $A$
    \begin{equation}
      det(A) = 5-(4+6j) = 1 - 6j
    \end{equation}
    finner $I_1$ vha. Cramers regel
    \begin{equation}
      I_1 = \frac{
        \left| \begin{array}{cc}
          0, & 2+3j \\
          6, & -5 \\
        \end{array} \right|}
      {det(A)} =
      \frac{-12-18j}{1-6j}\frac{(1+6j)}{(1+6j)} =
      \frac{96-90j}{37}
    \end{equation}
    finner $I_x$ vha. Cramers regel
    \begin{equation}
      I_x = \frac{
        \left| \begin{array}{cc}
          -1, & 0 \\
          2,  & 6 \\
        \end{array} \right|}
      {det(A)} =
      \frac{-6}{1-6j}\frac{(1+6j)}{(1+6j)} =
      \frac{-6-36j}{37}
    \end{equation}
    finner $V_0$ vha. Ohms lov
    \begin{equation}
      V=ZI\rightarrow V_0=I_1-4I_x = \frac{96-90j-4(-6-36j)}{37} = \frac{120+54j}{37}
    \end{equation}


  \newpage


  \section*{Oppgåve 9}
    Vi har tidlegare lært at superposisjon kun fungerer på kretsar med lineære kretselement.
    Sidan vi igjen får lineære likninger når vi omgjer frå tidsdomenet til det komplekse
    domenet antar eg at dette er grunnen for at superposisjon igjen fungerer som 
    kretsanalysemetode.

    \bigskip

    Finner bidrag frå spenningskjelda vha spenningsdeling
    \begin{equation}
      V_{0_B} = \frac{Z_1}{R_1 + R_2 + Z_1}V_s = \frac{-j}{25-j}12 = \frac{12-300j}{626}
    \end{equation}
    Finner bidrag frå straumkjelda vha. straumdeling
    \begin{equation}
      V_{0_A} = ZI = Z_1\left( \frac{R_1}{R_1+R_2+Z_1}i_s \right) =
      -j\left(\frac{24}{25-j}4\right) = \frac{-96+2400j}{626}
    \end{equation}
    finner $V_0$
    \begin{equation}
      V_0 = V_{0_A} + V_{0_B} = \frac{12-300j}{626} + \frac{-96+2400j}{626} =
      -\frac{42}{313}+\frac{1050}{313}j
    \end{equation}

  \section*{Oppgåve 10}
    Skriver alle fasevektorane over på kartesisk form. Gjer om spenningskjelda på venstre
    side til straumkjelde
    \begin{equation}
      I=\frac{V}{Z} \rightarrow I = 12
    \end{equation}
    Gjer om straumkjelda på høgre side til spenningskjelde
    \begin{equation}
      V=ZI \rightarrow V = (-j)4
    \end{equation}
    kretsen ser nå slik ut
    \begin{center}
      \begin{circuitikz}[american] \draw 
        (0, 0) to[I, l=12, invert] (0,3) -- (6,3)
               to[R, l=$1$] (9,3)
               to[C, l=$-j$] (12,3)
               to[V, l=$-4j$] (12,0) -- (0,0)
        (2,0)  to[R, l=$1$] (2,3)
        (4,0)  to[I, l=$2$] (4,3)
        (6,3)  to[R, l=$1$, i=$I_0$] (6,0)

               
        ;
      \end{circuitikz}
    \end{center}
    gjer spenningskjelda på høgre side tilbake til straumkjelde
    \begin{equation}
      I = \frac{V}{Z} \rightarrow I = \frac{-4j}{1-j} = 2-2j
    \end{equation}
    nå har vi tri straumkjelder i parallell som vi kan legge saman
    \begin{center}
      \begin{circuitikz}[american] \draw 
        (0, 0) to[I, l=12, invert] (0,3) -- (8,3)
               to[R, l=$1$] (8,1.5)
               to[C, l=$-j$] (8,0)
        (8,3) -- (10,3)
               to[I, l=$2-2j$, invert] (10,0) -- (0,0)
        (2,0)  to[R, l=$1$] (2,3)
        (4,0)  to[I, l=$2$] (4,3)
        (6,3)  to[R, l=$1$, i=$I_0$] (6,0)
        ;
      \end{circuitikz}
    \end{center}
    legger saman straumkjeldene
    \begin{equation}
      I = -12 + 2 + 2-2j = -8 -2j
    \end{equation}
    legger saman impedansane i dei to greinene vi ikkje bryr oss om
    \begin{equation}
      Z = 1||(1-j) = \frac{(1-j)}{1+1-j} = \frac{3-j}{5}
    \end{equation}
    kretsen ser nå slik ut
    \begin{center}
      \begin{circuitikz}[american] \draw 
        (0, 0) to[I, l=$-8-2j$] (0,3) -- (6,3) 
               to[R, l=$1$, i=$I_0$] (6,0) -- (0,0)
        (3, 0) to[generic, l=$3/5 -j/5$] (3,3)
        ;
      \end{circuitikz}
    \end{center}
    finner $I_0$ vha. straumdeling
    \begin{equation}
      I_0 = \frac{Z_2}{Z_1 + Z_2}I_s \rightarrow I_0 =
      \frac{\frac{3}{5} - \frac{1}{5}j}{1 + \frac{3}{5} - \frac{1}{5}j} (-8-2j) = -\frac{2}{13}(21+i)
      = -3,23 - 0,15 i
    \end{equation}

  \newpage

  \section*{Oppgåve 11}
    Nuller ut kjeldene og fjerner lasten for å finne $Z_{th}$
    \begin{center}
      \begin{circuitikz}[american] \draw 
        (0,0) -- (0,3)
              to[R, l=$1$] (4,3) -- (4,2)
        (4,0) -- (0,0)
        (2,0) to[C, l=$-j$] (2,2)
              to[R, l=$1$] (0,2)
        (2,2) -- (4,2)

        (4,2) to[short, -o] (5,2)
        (4,0) to[short, -o] (5,0)
        (5,2) to[open, v=$V_0$] (5,0)
        ;
      \end{circuitikz}
    \end{center}
    \begin{equation}
      Z_{th} = (-j||1)||1) = \frac{2}{5}-\frac{1}{5}j
    \end{equation}
    finner $V_{th}$ vha. nodespenning i supernode
    \begin{equation}
      \frac{V_0-4\angle \ang{0}}{1} +
      \frac{V_0-6\angle \ang{0} -4\angle \ang{0}}{1} +
      \frac{V_0-6\angle \ang{0}}{-j} - 2\angle \ang{0} = 0
      \rightarrow V_0 = \frac{16+j}{2+j} = \frac{38}{5} - \frac{4}{5}j
    \end{equation}
    \begin{center}
      \begin{circuitikz}[american] \draw 
        (0,0) to[V, l=$V_{th}$] (0,3)
              to[generic, l=$Z_{th}$, -*] (3,3)
              to[R, l=$R_{last}$, v=$V_0$, -*] (3,0) -- (0,0)
        ;
      \end{circuitikz}
    \end{center}
    finner $V_0$ vha. spenningsdeling
    \begin{equation}
      V_0 = \frac{1}{1 + 2/5 - j1/5}\left(\frac{38}{5}-\frac{4}{5}j\right) = \frac{38-4j}{7-j}
      = \frac{27+j}{5}
    \end{equation}

\end{document}
