\documentclass[12pt,a4paper]{article}
\title{%
  Øving 2 \\
  \large IELET1001 - Elektroteknikk \\
  }
\author{Gunnar Myhre, BIELEKTRO}

\usepackage[utf8]{inputenc}
\usepackage[norsk]{babel}
\usepackage[siunitx]{circuitikz}

\setlength\parindent{0pt}

\begin{document}
  \maketitle
    
  \section{Oppgåve 1}
    \subsection{a)}
      Det er tri vesentlege nodar i kretsen. Ein av nodene  på oversida av $R1$ og $R2$
      er trivielle sidan dei har likt elektrisk potensial.

    \subsection{b)}
      \begin{circuitikz}[american, scale=0.5] \draw
        (0,0) to[R, l=$R_{4}$, i=$i_4$] (7,0) -- (7,-2)
              to[R, *-, l=$R_2$, i=$i_2$] (7,-7) -- (5,-7)
              to[R, *-, l=$R_1$, i<=$i_1$] (5,-2) -- (7,-2)
        (0,0) -- (0,-2) to[R, *-, l=$R_3$, i=$i_3$] (5, -2) -- (7, -2)
        (0,-2) to[V, v=$V_s$] (0,-7) -- (5, -7)
        {[anchor=south east](7,-2) node {$v_1$}}
        ;
      \end{circuitikz}

    \subsection{c)}
      \begin{equation}
        V_s = 10V
      \end{equation}
      \begin{equation}
         R_1 = R_2 = R_3 = R_4 = 1k\si{\ohm}
      \end{equation}
      vi utfører spenningsdeling for å finne $v_1$
      \begin{equation}
        v_1 = \frac{R_1}{R_1+R_2}V_s = \frac{1}{2}10\si{\volt}=5\si{\volt}
      \end{equation}
      finner $i_{R4}$ ved Ohms lov
      \begin{equation}
         i_{R4}=\frac{5V}{1\si{\ohm}}=5A
      \end{equation}
      sidan $R_4 = R_3$ og dei står i parallell vil $i_{R4} = i_{R3}=5A$. Då
      veit vi også ved KCL at $i_{Vs}$ er $10A$.
      For å finne straumen $i_{R1}$ kan vi bruke formel for straumdeling
      \begin{equation}
        i_{R1} = \frac{G_1}{G_1 + G_2}10A=\frac{1}{2}10A=5A
      \end{equation}
      Dermed veit vi alle straumane og spenningane i kretsen.
  
      \begin{center}
      \begin{circuitikz}[american] \draw
        (0,0) to[R, l=$R_{4}$, i=5<\ampere>] (7,0) -- (7,-2)
              to[R, *-, l=$R_2$, i=5<\ampere>] (7,-7) -- (5,-7)
              to[R, *-, l=$R_1$, i<=5<\ampere>] (5,-2) -- (7,-2)
        (0,0) -- (0,-2) to[R, *-, l=$R_3$, i=5<\ampere>] (5, -2) -- (7, -2)
        (0,-2) to[V, v=10<\volt>, i=10<\ampere>] (0,-7) -- (5, -7)
        {[anchor=south east](7,-2) node {$5V$}}
        ;
      \end{circuitikz}
      \end{center}

  \section{Oppgåve 2}
    \subsection{a)}
      Effekten forbrukt av motstanden med $25k\si{\ohm}$ er avhengig av straumen
      som går igjennom maska. Vi veit at $P=Ri^{2}$, derfor er
      \begin{equation}
        2mW=25k\si{\ohm}\cdot i^{2} \rightarrow i^2 = \frac{2}{25}10^{-6}\si{\ampere}
        \rightarrow i = 0,28284\si{\milli\ampere}
      \end{equation}
        nå som vi veit straumen som skal til kan vi bruke ohms lov for å finne
        den ekvivalente resistansen
      \begin{equation}
        R_{ekv}=\frac{12\si{\volt}}{0,28284\si{\milli\ampere}}=42,426\si{\kilo\ohm}
      \end{equation}
        så finner vi $R$ ved å subtrahere dei andre motstandane sidan dei står i serie
      \begin{equation}
        R = 42,426\si{\kilo\ohm}-15\si{\kilo\ohm}-25\si{\kilo\ohm}=2,426\si{\kilo\ohm}
      \end{equation}

    \subsection{b)}
      For at kilden skal levere $3,6\si{\milli\watt}$ må $P=vi$ respekterast. Sidan
      $P$ og $v$ er definert må vi finne $i$
      \begin{equation}
        3,6\si{\milli\watt}=12\si{\volt}\cdot i
        \rightarrow i=\frac{3,6\cdot10^{-3}}{12}\si{\ampere}
        \rightarrow i=0,3\si{\milli\ampere}
      \end{equation}
      då kan vi finne $R_{ekv}$
      \begin{equation}
        v=Ri\rightarrow 12\si{\volt}=R_{ekv}\cdot0,3\si{\milli\ampere}
        \rightarrow R_{ekv}=\frac{12\si{\volt}}{0,3\si{\milli\ampere}}=40\si{\kilo\ohm}
      \end{equation}
      dermed er $R = 40\si{\kilo\ohm}-25\si{\kilo\ohm}-15\si{\kilo\ohm}=0\si{\kilo\ohm}$

    \section{Oppgåve 3}
      Ved KCL ser vi at $I_1 = 8\si{\milli\ampere}+4\si{\milli\ampere} = 12\si{\milli\ampere}$
      og $I_2 = 8\si{\milli\ampere}-2\si{\milli\ampere}=6\si{\milli\ampere}$

    \section{Oppgåve 4}
      Setter opp KCL i noden på venstresida og ser at
      \begin{equation}
        6\si{\milli\ampere}+3\si{\milli\ampere}-1,5I_x=0\rightarrow I_x = 6\si{\milli\ampere}
      \end{equation}
      så setter vi opp KCL i noden i midten og finner at
      \begin{equation}
        -6\si{\milli\ampere}+3I_x+I_1=0
        \rightarrow I_1 = 6\si{\milli\ampere}-18\si{\milli\ampere} = -12\si{\milli\ampere}
      \end{equation}
      $I_1$ er $-12\si{\milli\ampere}$

    \newpage

    \section{Oppgåve 5}
      Først kan vi slå sammen alle motstandane i serie sidan vi ikkje er ute etter spenningene
      mellom dei.

      \begin{center}
      \begin{circuitikz}[american] \draw
        (0,0) to[V, v=100<\volt>, invert] (0,4)
              to[R, l=40<\ohm>] (4,4)
        (0,0) to[R, l=60<\ohm>] (4,0)
              to[V, v=$V_2$, invert] (4,4)
        ;
      \end{circuitikz}
      \end{center}

      Med den opplyste effekten levert av $100V$-straumkilden kan vi finne straumen
      igjennom kretsen.

      \begin{equation}
        P=vi \rightarrow 200\si{\watt} =100\si{\volt}I\rightarrow I = 2\si{\ampere}
      \end{equation}
      Så kan vi finne $V_2$ vha. KVL
      \begin{equation}
        -100V +40\si{\ohm}I+V_2+60\si{\ohm}I=0\rightarrow V_2 = 100\si{\volt} -80V -120V = -100V
      \end{equation}

    \section{Oppgåve 6}
      Setter opp KVL i dei fire maskene. Først finner vi $V_x$
      \begin{equation}
        -V_x +12V - 8V = 0 \rightarrow V_x = 4V
      \end{equation}
      så finner vi $V_2$
      \begin{equation}
        4V_x +V_2 -12V = 0 \rightarrow V_2 = 12V-16V =-4V
      \end{equation}
      så finner vi $V_3$
      \begin{equation}
        -V_2 +V_3 -12V = 0 \rightarrow V_3 = 12V-4V=8V
      \end{equation}
      til slutt finner vi $V_1$
      \begin{equation}
        12V - V_1 +8V=0 \rightarrow -V_1 = -8V-12V=20V
      \end{equation}
       $V_1 = 20V$, $V_2 = -4V$ og $V_3 = 8V$

    \newpage

    \section{Oppgåve 7}
      Om $V_2 = 4V$ kan vi finne straumen $I_{V_2}=\frac{4V}{2\si{\ohm}}=2A$. Eg velger
      å løyse oppgåva vha. maskestraum.

      \begin{center}
        \begin{circuitikz}[american] \draw 
          (0,3) to [R,l_=5<\ohm>, i=$I_1$](0,0)
          (4,3) to [V,*-*, l^=24<\volt>] (4,0)
          (8,3) to [R,l=2<\ohm>, v=$V_2$] (8,0)
          (0,0) to[R, l=16<\ohm>, *-*] (4,0)
                to[R, l=2<\ohm>, *-*] (8,0)
          (0,3) to [R,  l^=3<\ohm>] (4,3)
                to [R, l^=4<\ohm>] (8,3)
          (0,0) to[R, l_=15<\ohm>] (0,-3)
                to[V, l_=$V_x$, *-*] (4,-3)
                to[R, l=2<\ohm>, *-*] (4,0)
          (4,-3) -- (8,-3)
                to[cI, l_=$2I_1$] (8,0)
          ;
          \draw[->,shift={(2,1.5)}] (120:.7cm) arc (120:-90:.7cm) node at(0,0){$I_a$};
          \draw[->,shift={(6,1.5)}] (120:.7cm) arc (120:-90:.7cm) node at(0,0){$I_b$};
          \draw[->,shift={(2,-1.5)}] (120:.7cm) arc (120:-90:.7cm) node at(0,0){$I_c$};
          \draw[->,shift={(6,-1.5)}] (120:.7cm) arc (120:-90:.7cm) node at(0,0){$I_d$};
        \end{circuitikz}
      \end{center}
      Setter opp fem likninger
      \begin{itemize}
        \item $33I_c -16I_a -2I_d -V_x = 0$ \hfill (KVL $I_c$)
        \item $I_b = 2A$ \hfill (oppgitt)
        \item $3I_a -2I_c = -3A$ \hfill (KVL $I_a$)
        \item $I_d = 2I_a$ \hfill (oppgitt v./avh. straumkilde og $I_1$)
        \item $4I_b - I_d = 12A$ \hfill (KVL $I_b$)
      \end{itemize}
      Løyser likningssettet som 5x5-matrise og får $I_a = -2A$, $I_b = 2A$,
      $I_c = -1,5A$, $I_d = -4A$ og $V_x = -9,5\si{\volt}$
        

    \newpage

    \section{Oppgåve 8}
      \begin{equation}
        9\si{\kilo\ohm}+5\si{\kilo\ohm}=14\si{\kilo\ohm}
      \end{equation}

      \begin{circuitikz}[scale=0.8] \draw
        (0,0) to[R, l=5<\kilo\ohm>, o-*] (3,0)
              to[R, l=3<\kilo\ohm>, *-*] (6,0)
              to[R, l=14<\kilo\ohm>, *-*] (9,-3) -- (6,-3)
              to[R, l=3<\kilo\ohm>, *-*] (6,0)
        (6,-3) -- (3,-3) to[R, l=18\si{\kilo\ohm}, *-*] (3,0)
        (3,-3) to[short, *-o] (0,-3)
        ;
      \end{circuitikz}

      \begin{equation}
        \frac{14\si{\kilo\ohm}\cdot3\si{\kilo\ohm}}{14\si{\kilo\ohm}+3\si{\kilo\ohm}}=
        \frac{42}{17}\si{\kilo\ohm}
      \end{equation}

      \begin{equation}
        \frac{42}{17}\si{\kilo\ohm}+3\si{\kilo\ohm} = \frac{93}{17}\si{\kilo\ohm}
      \end{equation}

      \begin{circuitikz}[scale=0.8] \draw
        (0,0) to[R, l=5<\kilo\ohm>, o-*] (3,0)
              to[R, l=$\frac{93}{17}\si{\kilo\ohm}$, *-*] (6,-3)
        (6,-3) -- (3,-3) to[R, l=18\si{\kilo\ohm}, *-*] (3,0)
        (3,-3)to[short, *-o] (0,-3)
        ;
      \end{circuitikz}

      \begin{equation}
        \frac{\frac{93}{17}\si{\kilo\ohm}\cdot 18\si{\kilo\ohm}}
        {\frac{93}{17}\si{\kilo\ohm} + 18\si{\kilo\ohm}}+5\si{\kilo\ohm}
        = \frac{1223}{133}\si{\kilo\ohm} \approx 9,195 \si{\kilo\ohm}
      \end{equation}
      $R_{AB} = 9,2 \si{\kilo\ohm}$

    \section{Oppgåve 9}
      Først finner vi $R_{ekv}$
      \begin{equation}
        R_{ekv}=\left( \frac{12\cdot6}{12+6}+2 \right) \si{\kilo\ohm}
        \rightarrow R_{ekv}= 6\si{\kilo\ohm}
      \end{equation}
      så kan vi finne straumen ut ifrå spenningskilden vha. Ohms lov
      \begin{equation}
        I_s=\frac{12\si{\volt}}{6\si{\kilo\ohm}}\rightarrow I_s=2\si{\milli\ampere}
      \end{equation}
      nå kan vi finne straumen $I_1$ vha. straumdeling
      \begin{equation}
        I_1 = \frac{1/6}{1/12+1/6}2\si{\milli\ampere}
        \rightarrow I_1 = \frac{4}{3}\si{\milli\ampere}=1,33\si{\milli\ampere}
      \end{equation}
      nå som vi veit straumane kan vi rekne ut spenningsfalla over $2k$- og $8k$-motstandane
      slik at vi kan finne $V_0$
      \begin{equation}
        v_{2k}=2\si{\kilo\ohm}\cdot2\si{\milli\ampere}\rightarrow v_{2k}=4\si{\volt}
      \end{equation}
      straumen igjennom $R_{8k}$ er $(2-\frac{4}{3})\si{\milli\ampere}
      =\frac{2}{3}\si{\milli\ampere}$
      \begin{equation}
        v_{8k}=8\si{\kilo\ohm}\cdot0,66\si{\milli\ampere}=\frac{16}{3}\si{\volt}
      \end{equation}
      $V_0=12\si{\volt}-4\si{\volt}-\frac{16}{3}\si{\volt} = \frac{8}{3}\si{\volt}=2,67\si{\volt}$

    \section{Oppgåve 10}
      Her kan vi med ein gong sjå at spenningsfallet over $8k$-motstanden vil vere
      $8V$ (sidan spenningsfallet over $4k$-motstanden er $4V$ og motstandane står
      i serie). Dermed er kildespenninga $V_S = 12V$.

    \section{Oppgåve 11}
      Ingen av motstandane er i utgangspunktet i serie eller i parallell, men
      "mercedesstjerna" av $2\si{\ohm}$-motstandar kan vi omgjere vha. Y-Delta–metoden.
      Sidan alle dei tri motstandane har lik resistans vil vi kun trenge éi likning:
      \begin{equation}
        R = \frac{2\cdot2 + 2\cdot2 + 2\cdot2}{2}\si{\ohm}=6\si{\ohm}
      \end{equation}
      den nye kretsen ser slik ut

      \begin{center}
        \begin{circuitikz}[american, scale=0.8] \draw
          (0,0) to[short, o-*] (3,0)
                to[R, l=5<\ohm>, *-*] (8,0) -- (8,-2)
          (3,0) -- (3,-2) -- (4,-2) to[R, l=6<\ohm>, *-*] (7,-2) -- (8,-2)
                to[R, l=10<\ohm>, *-*] (8, -5)
          (3,-2)to[R, l_=10<\ohm>, *-*] (3,-5)
          (0,-5)to[short, o-*] (8,-5)
          (4,-2)to[R, l=6<\ohm>, *-*] (4,-5)
          (7,-2)to[R, l_=6<\ohm>, *-*] (7,-5)
          ;
        \end{circuitikz}
      \end{center}
      slår saman dei parallelle motstandane $\frac{5\cdot6}{5+6} = 2,727$,
      $\frac{6\cdot10}{6+10}=3,75$
      \begin{equation}
        R_{xy}=\frac{3,75\cdot(2,727+3,75)}{3,75+(2,727+3,75)}\si{\ohm}
        \rightarrow R_{xy} =2,375\si{\ohm} 
      \end{equation}

    \section{Oppgåve 12}
      Som i førige oppgåve står ingen av motstandane i utgangspunktet i parallell eller i
      serie, så vi må ty til Delta-Y–metoden for å finne $R_{ekv}$. 
      Eg velger først å transformere Y-en $abc$ rundt noda $n$ og får ein Delta med $def$
      \begin{equation}
        R_d = \frac{10\cdot20+20\cdot5+10\cdot5}{10}\si{\ohm}=35\si{\ohm}
      \end{equation}
      \begin{equation}
        R_e = \frac{10\cdot20+20\cdot5+10\cdot5}{20}\si{\ohm}=17,5\si{\ohm}
      \end{equation}
      \begin{equation}
        R_f = \frac{10\cdot20+20\cdot5+10\cdot5}{5}\si{\ohm}=70\si{\ohm}
      \end{equation}

      I denne nye kretsen står $R_f$ i parallell med $30\si{\ohm}$

      \begin{center}
        \begin{circuitikz}[american, scale=0.8] \draw
          (0,0)  to[V, v=120<\volt>, *-*, invert] (0,6) -- (2,6)
                 to[R, l_=12.5<\ohm>, *-*] (2, 3)
                 to[R, l_=15<\ohm>, *-*] (2, 0) -- (0,0)
          (2,3)  to[R, l_=17.5<\ohm>, *-*] (4,6)
          (2,3)  to[R, l=35<\ohm>, *-*] (4,0)
          (4,6)  to[R, l=70<\ohm>, *-*] (4,0)
          (2,6) -- (6,6) to[R, l=30<\ohm>, *-*] (6,0) -- (2,0)
          ;
        \end{circuitikz}
      \end{center}

      Etter å ha slått saman $\frac{70\cdot30}{70+30}\si{\ohm}=21\si{\ohm}$ transformerer
      vi tilbake

      \begin{equation}
        R_d = \frac{17,5\cdot21}{21+35+17,5}\si{\ohm}=5\si{\ohm}
      \end{equation}
      \begin{equation}
        R_e = \frac{21\cdot35}{21+35+17,5}\si{\ohm}=10\si{\ohm}
      \end{equation}
      \begin{equation}
        R_f = \frac{35\cdot17,5}{21+35+17,5}\si{\ohm}=\frac{25}{3}\si{\ohm}\approx 8,33\si{\ohm}
      \end{equation}

      \begin{center}
        \begin{circuitikz}[american, scale=0.8] \draw
          (0,0)  to[V, v=120<\volt>, *-*, invert] (0,6) -- (4,6)
                 to[R, l_=12.5<\ohm>, *-*] (2, 3)
                 to[R, l_=15<\ohm>, *-*] (4, 0) -- (0,0)
          (4,0)  to[R, l_=10<\ohm>, *-*] (6,3)
                 to[R, l_=5<\ohm>, *-*] (4,6)
          (2,3)  to[R, l=8.33<\ohm>, *-*] (6,3)
          ;
        \end{circuitikz}
      \end{center}

      transponerer den nederste trikanten med Y-Delta
      \begin{equation}
        R_d = \frac{8,33\cdot10}{8,33+10+15}\si{\ohm}=2,5\si{\ohm}
      \end{equation}
      \begin{equation}
        R_e = \frac{10\cdot15}{8,33+10+15}\si{\ohm}=4,5\si{\ohm}
      \end{equation}
      \begin{equation}
        R_f = \frac{15\cdot8,33}{8,33+10+15}\si{\ohm}=3,75\si{\ohm}
      \end{equation}
        
      \begin{center}
        \begin{circuitikz}[american, scale=0.8] \draw
          (0,0)  to[V, v=120<\volt>, *-*, invert] (0,6) -- (2,6)
                 to[R, l_=12.5<\ohm>, *-*] (2, 3)
                 to[R, l=3.75<\ohm>, *-*] (5, 3)
                 to[R, l=2.5<\ohm>, *-*] (8, 3)
                 to[R, l=5<\ohm>, *-*] (8,6) -- (2,6)
          (0,0)  -- (5, 0)
                 to[R, l=4.5<\ohm>, *-*] (5,3)
          ;
        \end{circuitikz}
      \end{center}
      Nå står alle motstandane i serie eller parallell, så vi kan rekne ut $R_{ekv}$ på
      vanleg vis.
      \begin{equation}
        R_{ekv}=\left( \frac{16,25\cdot7,5}{16,25+7,5}+4,5 \right)\si{\ohm}=(5,1315+4,5)\si{\ohm}
        \rightarrow R_{ekv}=9,63\si{\ohm}
      \end{equation}
      Ohms lov viser at straumen ut ifrå spenningskilden er
      \begin{equation}
        I = \frac{120\si{\volt}}{9,63\si{\ohm}} = 12,46A
      \end{equation}

\end{document}
