\documentclass[12pt,a4paper]{article}
\title{%
  Øving 5 \\
  \large IELET1001 - Elektroteknikk \\
  }
\author{Gunnar Myhre, BIELEKTRO}

\usepackage[utf8]{inputenc}
\usepackage[norsk]{babel}
\usepackage[siunitx]{circuitikz}
\usepackage{amsmath}

\setlength\parindent{0pt}

\begin{document}
  \maketitle
    
  \section{Oppgåve 1}
    Løyser med superposisjon. Finner bidraget frå spenningskjelda:
    \begin{center}
      \begin{circuitikz}[american] \draw
        (0,0)  to[R, l=8<\ohm>] (3,0)
               to[R, l=4<\ohm>, v=$V$] (3, -3) -- (0, -3)
               to[V, l=6<\volt>, invert] (0, 0)
        ;
      \end{circuitikz}
    \end{center}
    \begin{equation}
      V_{vs} = \frac{4}{8+4}V_s = 2V
    \end{equation}
    Finner bidraget frå straumkjelda:
    \begin{center}
      \begin{circuitikz}[american] \draw
        (0,0)  to[R, l=8<\ohm>] (3,0)
               to[R, l=4<\ohm>, v=$V$] (3, -3) -- (0, -3) -- (0, 0)
        (3,0)  -- (5, 0)
               to[I, l=3<\ampere>, invert] (5, -3) -- (3, -3)

        ;
      \end{circuitikz}
    \end{center}
    kjeldetransformerer
    \begin{center}
      \begin{circuitikz}[american] \draw
        (0,0)  to[R, l=8<\ohm>] (3,0)
               to[V, l=24<\volt>] (3, -3) -- (0, -3)
        (0,0)  to[R, l=4<\ohm>, v=$V$] (0, -3)
        ;
      \end{circuitikz}
    \end{center}
    \begin{equation}
      V_{is} = \frac{4}{8+4}24V = 8V
    \end{equation}
    finner $V_o$
    \begin{equation}
      V_{o} = V_{is} + V_{vs} = 2V + 8V = 10V
    \end{equation}

  \section{Oppgåve 2}
    Finner bidraget frå spenningskjelda:
    \begin{center}
      \begin{circuitikz}[american] \draw
        (0,0)  to[R, l=4<\kilo\ohm>, v=$V$] (3,0)
               to[R, l=8<\kilo\ohm>] (3, -3) -- (0, -3)
               to[V, l=8<\volt>, invert] (0, 0) -- (-3, 0)
               to[R, l=8<\kilo\ohm>] (-3, -3) -- (0, -3)
        ;
      \end{circuitikz}
    \end{center}
    \begin{equation}
      V_{vs} = \frac{4}{4+8}8V = \frac{8}{3}V
    \end{equation}
    Finner bidraget frå straumkjelda:
    \begin{center}
      \begin{circuitikz}[american] \draw
        (0,0)  -- (3,0)
               to[R, l=8<\kilo\ohm>] (3, -3) -- (0, -3)
               to[I, l=4<\milli\ampere>] (0, 0)
               to[R, l=4<\kilo\ohm>, v<=$V$] (-3, 0) -- (-3, -3) -- (0, -3)
        ;
      \end{circuitikz}
    \end{center}
    kjeldetransformerer $4mA$ og $8\si{\kilo\ohm}$ til $32V$
    \begin{equation}
      V_{is} = -\frac{4}{4+8}32V = -\frac{32}{3}V
    \end{equation}
    finner $V$:
    \begin{equation}
      V = V_{is} + V_{vs} = \frac{8}{3}V - \frac{32}{3}V = 8V
    \end{equation}

  \section{Oppgåve 3}
    Finner bidrag frå $8V$-kjelda:
    \begin{center}
      \begin{circuitikz}[american] \draw
        (0,0)  to[V, l=8<\volt>, invert] (0, 3) --(3, 3)
               to[R, l=4<\kilo\ohm>, v=$V_o$] (3, 0)
        (3,3) -- (6,3) -- (6,0)
               to[R, l=4<\kilo\ohm>] (3, 0)
               to[R, l=4<\kilo\ohm>] (0, 0)
        (6,0) -- (6,-3)
               to[R, l=8<\kilo\ohm>] (0,-3) -- (0,0)
        ;
      \end{circuitikz}
    \end{center}
    bruker Y-Delta for å forenkle kretsen
    \begin{center}
      \begin{circuitikz}[american] \draw
        (0,0)  to[V, l=8<\volt>, invert] (0, 3) --(3, 3)
               to[R, l=4<\kilo\ohm>, v=$V_o$] (3, 0)
               to[R, l=1<\kilo\ohm>] (3, -3)
        (3,3) -- (6,3) -- (6,-3)
               to[R, l=2<\kilo\ohm>] (3,-3)
               to[R, l=2<\kilo\ohm>] (0,-3) -- (0,0)
        (3,-3) node[label={[font=\footnotesize]150:$v$}] {}
        ;
      \end{circuitikz}
    \end{center}
    \begin{equation}
      KVL_v: \frac{v}{2} + \frac{v-8}{2} + \frac{v-8}{5} = 0
      \Rightarrow v = \frac{14}{3}V
    \end{equation}
    spenningsdeling
    \begin{equation}
      V_{o1} = \frac{4}{5} \frac{10}{3}V = \frac{8}{3}V
    \end{equation}
    Finner bidrag frå $4V$-kjelda:
    \begin{center}
      \begin{circuitikz}[american] \draw
        (0,0) -- (0, 3) --(3, 3)
               to[R, l=4<\kilo\ohm>, v=$V_o$] (3, 0)
        (3,3) -- (6,3)
               to[V, l=4<\volt>, invert] (6,0)
               to[R, l=4<\kilo\ohm>] (3, 0)
               to[R, l=4<\kilo\ohm>] (0, 0)
        (6,0) -- (6,-3)
               to[R, l=8<\kilo\ohm>] (0,-3) -- (0,0)
        ;
      \end{circuitikz}
    \end{center}
    bruker Y-Delta for å forenkle kretsen
    \begin{center}
      \begin{circuitikz}[american] \draw
        (0,0) -- (0, 3) --(3, 3)
               to[R, l=4<\kilo\ohm>, v=$V_o$] (3, 0)
               to[R, l=1<\kilo\ohm>] (3, -3)
        (3,3) -- (6,3)
               to[V, l=4<\volt>, invert] (6,-3)
               to[R, l=2<\kilo\ohm>] (3,-3)
               to[R, l=2<\kilo\ohm>] (0,-3) -- (0,0)
        (3,-3) node[label={[font=\footnotesize]150:$v$}] {}
        ;
      \end{circuitikz}
    \end{center}
    \begin{equation}
      KVL_v: \frac{v-4}{2} + \frac{v}{5} + \frac{v}{2} = 0
      \Rightarrow v = \frac{5}{3}V
    \end{equation}
    spenningsdeling
    \begin{equation}
      V_{o2} = -\frac{4}{5} \frac{5}{3}V = -\frac{4}{3}V
    \end{equation}
    Finner bidrag frå $4mA$-kjelda:
    \begin{center}
      \begin{circuitikz}[american] \draw
        (0,0) -- (0, 3) --(3, 3)
               to[R, l=4<\kilo\ohm>, v=$V_o$] (3, 0)
        (3,3) -- (6,3) -- (6,0)
               to[R, l=4<\kilo\ohm>] (3, 0)
               to[R, l=4<\kilo\ohm>] (0, 0)
        (6,0) -- (6,-3)
               to[R, l=4<\kilo\ohm>] (3, -3)
               to[R, l=4<\kilo\ohm>] (0, -3) -- (0, 0)
        (3,-3) to[I, l=4<\milli\ampere>] (3, 0)
        ;
      \end{circuitikz}
    \end{center}
    her kan vi sjå dei tre øverste 4k-motstandane står i parallell ut ifrå
    4mA-kjelda. M.a.o. vil det gå $\frac{4mA}{3}$ igjennom kvar av dei.\\
    Derfor vil $V_{o3} = -4\si{\kilo\ohm}\frac{4}{3}mA = \frac{16}{3}V$
    \begin{equation}
      V_o = V_{o1} + V_{o2} + V_{o3} = \frac{8}{3}V - \frac{4}{3}V - \frac{16}{3}V = -4V
    \end{equation}

  \section{Oppgåve 4}
    Finner $R_{th}$ ved å fjerne lasten og nulle ut kjelda (kortslutte)
    \begin{equation}
      R_{th} = \left(10 + \frac{10\cdot20}{10+20} \right) \si{\ohm} = \frac{50}{3}\si{\ohm}
    \end{equation}
    finner $V_{th}$ ved å fjerne lasten, spenningsdeling:
    \begin{equation}
      V_{th} = \frac{28}{3}
    \end{equation}
    finner effekten brukt i $5\si{\ohm}$-motstanden
    \begin{equation}
      P_{last}=\left( \frac{v_{th}}{R_l + R_{th}} \right) ^2 R_L
      = \left( \frac{\frac{28}{3}V}{5\si{\ohm} + \frac{50}{3}\si{\ohm}} \right) ^2 5 \si{\ohm}
      = 928 mW
    \end{equation}

  \section{Oppgåve 5}
    Sidan det ikkje er nokon uavhengige kjelder i kretsen veit vi at $V_{th}$ vil bli 0.
    For å finne $R_{th}$ setter vi opp spenningskjelda $V_T$ mellom dei åpne terminalane.
    \begin{center}
      \begin{circuitikz}[american] \draw
        (0,0)  -- (2,0) -- (4, 0)
               to[R, l=1<\kilo\ohm>] (7, 0) -- (9, 0)
               to[V, l=$V_T$, i=$I_T$, o-o] (9,-3) -- (7,-3)
               to[R, l=1<\kilo\ohm>] (7, 0)
        (7,-3) -- (4,-3)
               to[R, l=1<\kilo\ohm>, i<=$I_x$] (4, 0)
        (4,-3) -- (2,-3) -- (0,-3)
               to[cI, l=$2I_x$, invert] (0,0)
        (2,-3) to[R, l=1<\kilo\ohm>] (2, 0)

        (3,0)  node[label={[font=\footnotesize]150:$v$}] {}
        (3,-3) node[ground]{}
               ;
      \end{circuitikz}
    \end{center}
    Bruker nodespenningsmetoden. Setter opp KCL i node $v$
    \begin{equation}
      2I_x + \frac{v}{1\si{\kilo\ohm}}+ I_x + \frac{v-V_T}{1\si{\kilo\ohm}} = 0
      \Rightarrow V_T = 2v + 3I_x\si{\kilo\ohm}
    \end{equation}
    vi kan finne $v$ ved $I_x$:
    \begin{equation}
      v = I_x\si{\kilo\ohm}
    \end{equation}
    Setter inn og finner forholdet mellom $V_T$ og $v$
    \begin{equation}
      V_t = 5v   \Rightarrow v = \frac{1}{5}V_T
    \end{equation}
    KCL i noden $V_T^+$:
    \begin{equation}
      -I_T + \frac{V_T}{1\si{\kilo\ohm}} + \frac{V_T - \frac{1}{5}V_T}{1\si{\kilo\ohm}} = 0
      \Rightarrow V_T = I_T\frac{5}{9}\si{\kilo\ohm}
    \end{equation}
    setter inn i generell formel for Thevenin
    \begin{equation}
      V_T = R_{th}I_T + V_{th} \Rightarrow R_{th}I_T = I_T\frac{5}{9}\si{\kilo\ohm}
      \Rightarrow R_{th}=\frac{5}{9}\si{\kilo\ohm}
    \end{equation}
    $V_{th} = 0$, $R_{th} = \frac{5}{9}\si{\kilo\ohm} = 555.6\si{\ohm}$

  \section{Oppgåve 6}
    Nuller ut kjeldene og finner $R_{N} = \frac{20\cdot5}{25} = 4\si{\ohm}$. Kortslutter terminalane
    a og b for å finne $I_N$. Kjeldetransformerer 12V-kjelda.
    \begin{center}
      \begin{circuitikz}[american] \draw
        (-3,0) to[I, l=2<\ampere>, invert] (-3,-3) -- (0, -3)
               to[I, l=3<\ampere>] (0,0) -- (-3,0)
        (0,0) -- (3,0)
               to[R, l=4<\ohm>] (3,-3)
        (0,-3) -- (3,-3)
               to[R, l=8<\ohm>] (6,-3) -- (9,-3) -- (9,0) -- (6,0)
               to[R, l=5<\ohm>] (6,-3)
        (3,0)  to[R, l=8<\ohm>] (6,0)

               ;
      \end{circuitikz}
    \end{center}
    slår saman kjeldene til $5A$ og kjeldetransformerer igjen til $20V$
    \begin{center}
      \begin{circuitikz}[american] \draw
        (3,0)  to[V, l=20<\volt>] (3,-3)
               to[R, l=8<\ohm>] (6,-3)
               to[R, l=5<\ohm>] (6,0)
        (3,0)  to[R, l=12<\ohm>] (6,0) -- (8,0) -- (8,-3) -- (6,-3)
               ;
      \end{circuitikz}
    \end{center}
    spenninga over $12\si{\ohm}$-motstanden er
    \begin{equation}
      \frac{12}{20}20V = 12V
    \end{equation}
    dermed er straumen igjennom $12\si{\ohm}$-motstanden $1A$. Det går ingen straum igjennom
    $5\si{\ohm}$-motstanden pga. kortstlutninga, derfor er $I_N = 1A$

  \section{Oppgåve 7}
    Ser først etter $R_N$. Om vi nuller ut den uavhengige spenningskjelda vil den kortslutte stien
    som $4\si{\ohm}$-motstanden står i, dermed vil $I_x$ verte null, og den avhengige straumkjelda
    vil heller ikkje ha nokon påverknad på $R_N$. Derfor er $R_N = 5\si{\ohm}$
    \begin{center}
      \begin{circuitikz}[american] \draw
        (0,0)  to[R, l=4<\ohm>, i=$I_x$] (0,-3) -- (3,-3)
               to[V, l=10<\volt>, invert] (3,0) -- (0,0)
        (3,-3) -- (6,-3)
               to[short, i<=$I_N$] (6,0) -- (6,2)
               to[cI, l=$2I_x$, invert] (3,2) -- (3,0)
               to[R, l=5<\ohm>] (6,0)
               ;
      \end{circuitikz}
    \end{center}
    \begin{equation}
      KVL_x: 4I_x = 10
    \end{equation}
    \begin{equation}
      KVL_N: -10 +5I_N -10I_x =0
    \end{equation}
    slår saman likningene
    \begin{equation}
      5I_N - 10\frac{10}{4} = 10 \Rightarrow 5I_N = 10+25
      \Rightarrow I_N = \frac{35}{5} = 7
    \end{equation}
    $I_N = 7A$, $R_N = 5\si{\ohm}$

  \section{Oppgåve 8}
    Først kan vi slå saman alle straumkjeldene sidan dei står i parallell, $I = 7A$. Så kan vi transformere
    denne kjelda med $6\si{\ohm}$ som står i serie, vi får ein ny spenningskjelde $42V$.
    \begin{center}
      \begin{circuitikz}[american] \draw
        (0,0) to[R, l=6<\ohm>] (3,0)
              to[R, l=10<\ohm>, v=$V$] (3,-3) -- (0,-3)
              to[V, l=42<\volt>, invert] (0,0)
               ;
      \end{circuitikz}
    \end{center}
    Nå kan vi finne $V$ ved spenningsdeling
    \begin{equation}
      V = 42 - \frac{6}{16}42V = 26,25V
    \end{equation}

  \section{Oppgåve 9}
    Motstandane på venstre side står i parallell med den eine 12k-motstanden i midten. Eg kjeldetransformerer
    spenningskjelda på høgre side og slår saman 4k med 12k. Definerer $I_A$ som straumen igjennom 3k-motstanden
    \begin{center}
      \begin{circuitikz}[american] \draw
        (0,0)  to[I, l=2<\milli\ampere>] (0,-3) -- (3,-3)
               to[R, l=4<\kilo\ohm>] (3,0) -- (0,0)
        (3,-3) -- (6, -3)
               to[R, l=3<\kilo\ohm>] (6,0) -- (9,0)
               to[I, l=1.5<\milli\ampere>] (9,-3) -- (6,-3)
        (3,0)  to[R, l=3<\kilo\ohm>, v=$V_o$, i=$I_A$] (6,0)
               ;
      \end{circuitikz}
    \end{center}
    gjør KVL i den midterste maska
    \begin{equation}
      10I_A + 8mA -4,5mA = 0 \Rightarrow I_A = -\frac{3,5mA}{10} = -0,35mA
    \end{equation}
    ved ohms lov finner vi spenningsfallet over 3k-motstanden
    \begin{equation}
      V_o = -0,35mA \cdot 3\si{\kilo\ohm} = -1,05V
    \end{equation}

  \section{Oppgåve 10}
    Velger å bruke Theveninekvivalent.
    Kjeldetransformerer 12V til 2A. $6\si{\ohm}$ og $12\si{\ohm}$ vert $4\si{\ohm}$.
    \begin{center}
      \begin{circuitikz}[american] \draw
        (0,3) -- (-2,3)
               to[I, l=2<\ampere>, invert] (-2,0) --(3,0)
        (3,3)  to[R, l=3<\ohm>] (0,3)
               to[R, l=4<\ohm>] (0,0) -- (3,0)
               to[I, l=2<\ampere>] (3,3)
               to[R, l=2<\ohm>, -o, i=$I$] (6,3)
        (3,0)  to[short, -o] (6,0)
        (3,3)  node[label={[font=\footnotesize]150:$v$}] {}
               ;
      \end{circuitikz}
    \end{center}
    Kjeldetransformerer straumkjelda igjen og får 8V. $4\si{\ohm}$ står i serie med $3\si{\ohm}$,
    til slutt står vi igjen med denne kretsen: 
    \begin{center}
      \begin{circuitikz}[american] \draw
        (3,3)  to[R, l=7<\ohm>] (0,3)
               to[V, l=8<\volt>] (0,0) -- (3,0)
               to[I, l=2<\ampere>] (3,3)
               to[R, l=2<\ohm>, -o, i=$I$] (6,3)
        (3,0)  to[short, -o] (6,0)
        (3,3)  node[label={[font=\footnotesize]150:$v$}] {}
               ;
      \end{circuitikz}
    \end{center}
    gjør KCL i node v.
    \begin{equation}
      -2A + \frac{v-8V}{7\si{\ohm}} = 0 \Rightarrow v = 22V
    \end{equation}
    spenninga i denne noda er også $V_{th}$ sidan $I = 0$. Vi nuller ut kjeldene og
    ser at $R_{th} = 9\si{\ohm}$
    \begin{equation}
      P_{max} = \left( \frac{22V}{9+9} \right) ^2 \cdot 9\si{\ohm} = 13,44W
    \end{equation}

  \section{Oppgåve 11}
    Leiter først opp $V_{oc} = V_{th}$ ved å finne spenninga når kretsen er åpen
    mellom terminalane A og B
    \begin{center}
      \begin{circuitikz}[american] \draw
        (0,0)  to[R, l=8<\ohm>, v<=$V_A$] (3,0)
               to[R, l=4<\ohm>, *-] (6,0)
               to[cV, l=$2V_A$, *-o, invert] (9,0)
        (0,0)  to[V, l=12<\volt>] (0,-3) -- (6,-3)
               to[I, l=2<\ampere>] (6,0)
        (6,-3)  to[short, -o] (9,-3)
        (3,0)  node[label={[font=\footnotesize]150:$v_1$}] {}
        (6,0)  node[label={[font=\footnotesize]150:$v_2$}] {}
               ;
      \end{circuitikz}
    \end{center}

    \begin{equation}
      KCL_1: \frac{v_1 - 12V}{8\si{\ohm}} + \frac{v_1 - v_2}{4\si{\ohm}} = 0
      \rightarrow 3v_1 - 2V_2 = 12V
    \end{equation}
    \begin{equation}
      KCL_2: \frac{v_2 - v_1}{4\si{\ohm}} -2A = 0
      \rightarrow v_2 - v_1 = 8V
    \end{equation}
    Nå kjenner vi $v_1$ og $v_2$. Finner $V_A$
    \begin{equation}
      V_A = v_1 - 12V = 16V
    \end{equation}
    finner $V_{oc}$
    \begin{equation}
      V_{oc} = v_2 + 2V_A = 36V + 2\cdot16V = 68V
    \end{equation}
    Nå veit vi at $V_{th} = 68V$, finner $R_{th}$ ved å nulle ut dei uavhengige kjeldene
    \begin{center}
      \begin{circuitikz}[american] \draw
        (0,0)  to[R, l=8<\ohm>, v<=$V_A$] (3,0)
               to[R, l=4<\ohm>, *-] (6,0)
               to[cV, l=$2V_A$, *-o, invert] (9,0)
        (0,0)  to[short] (0,-3) -- (6,-3)
               to[short, -*] (6,-2.5)
        (6,0)  to[short, -*] (6,-0.5)
        (6,-3)  to[short, -o] (9,-3)

        (9,0)  to[V, l=$V_T$, i=$I_T$] (9,-3)
        (3,0)  node[label={[font=\footnotesize]150:$v_1$}] {}
        (6,0)  node[label={[font=\footnotesize]150:$v_2$}] {}
               ;
      \end{circuitikz}
    \end{center}
    leiter etter $V_T$ og $I_T$
    \begin{equation}
      KCL_1: \frac{V_A}{8\si{\ohm}} + \frac{V_A-(V_T-2V_A)}{4\si{\ohm}} = 0
      \rightarrow V_T = \frac{7}{2}V_A
    \end{equation}
    $I_T$ går igjennom heile kretsen, så vi kan ta utgongspunkt i $8\si{\ohm}$-motstanden
    \begin{equation}
      I_T = \frac{V_A}{8\si{\ohm}}
    \end{equation}
    finner $R_{th} = \frac{V_T}{I_T} = \frac{7 \cdot 8}{2}\si{\ohm} = 28\si{\ohm}$. Nå
    har vi nok informasjon til å rekne ut effekten.
    \begin{equation}
      P_{max} = \frac{V_{th}^2}{4R_{th}} = \frac{68V^2}{4\cdot28\si{\ohm}} = 41,28W
    \end{equation}

\end{document}
