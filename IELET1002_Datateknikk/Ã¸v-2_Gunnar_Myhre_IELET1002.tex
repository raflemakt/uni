\documentclass[12pt,a4paper]{article}
\title{%
  Øving 2 \\
  \large IELET1002 - Datateknikk \\
  }
\author{Gunnar Myhre, BIELEKTRO}

\usepackage[utf8]{inputenc}
\usepackage[norsk]{babel}
\usepackage[siunitx]{circuitikz}
\usepackage{karnaugh-map}

\newcommand{\N}{\overline}

\setlength\parindent{0pt}

\begin{document}
  \maketitle
    
  \section{Oppgåve 1}
    \subsection{a)}
      Bruker reknereglar T5a, T5b, P4a
      \begin{equation}
        T=\N{A + BC}=\N{A}\cdot\N{BC}=\N{A}(\N{B}+\N{C})
        = \bar{A}\bar{B} + \bar{A}\bar{C}
      \end{equation}
    \subsection{b)}
      T5a
      \begin{equation}
        T=\N{AB+\bar{A}\bar{B}} = \N{AB} \cdot \N{\bar{A}\bar{B}}
      \end{equation}
      T5b, P4
      \begin{equation}
        (\bar{A}+\bar{B})(A+B) = A(\bar{A}+\bar{B}) + B(\bar{A}+\bar{B})
      \end{equation}
      P4, P5
      \begin{equation}
        A\bar{A}+A\bar{B}+B\bar{A}+B\bar{B} = A\bar{B} + \bar{A}B
      \end{equation}
    \subsection{c)}
      T5b
      \begin{equation}
        T=\N{(A+\bar{B})(\bar{B}+C)} = \N{(A+\bar{B})}+\N{(\bar{B}+C)}
      \end{equation}
      T5a
      \begin{equation}
        \bar{A}B+B\bar{C}
      \end{equation}
    
  \section{Oppgåve 2}
    \subsection{a)}
      Setter opp funksjonstabell for uttrykket $F(A,B,C) = (\bar{A}+B)(\bar{B}+A)$. 
      Uttrykket er allereie nesten på PaSS-form, vi kan legge til begge variantar
      av den tredje variablen i kvar av summane. (Eks.: $(\bar{A}+B)\rightarrow
      (\bar{A}+B+C)(\bar{A}+B+\bar{C})$)
      \begin{center}
        \begin{tabular}{ |c|c|c|c|c| }
          \hline
          Indeks & A & B & C & F(A,B,C) \\
          \hline
          0 & $0$ & $0$ & $0$ & 1 \\
          \hline
          1 & $0$ & $0$ & $1$ & 1 \\
          \hline
          2 & $0$ & $1$ & $0$ & 0 \\
          \hline
          3 & $0$ & $1$ & $1$ & 1 \\
          \hline
          4 & $1$ & $0$ & $0$ & 0 \\
          \hline
          5 & $1$ & $0$ & $1$ & 0 \\
          \hline
          6 & $1$ & $1$ & $0$ & 0 \\
          \hline
          7 & $1$ & $1$ & $1$ & 1 \\
          \hline
        \end{tabular}
      \end{center}
      Produkt av standardsumform:
      \begin{equation}
         F(A,B,C) = (\bar{A}+B+C)(\bar{A}+B+\bar{C})(A+\bar{B}+C)(\bar{A}+\bar{B}+C)
      \end{equation}
      Sum av standardproduktform:
      \begin{equation}
        F(A,B,C) = \bar{A}\bar{B}\bar{C}+\bar{A}\bar{B}C+\bar{A}BC+ABC
      \end{equation}
      Indeksformer:
      \begin{equation}
        F(A,B,C) = \Sigma(0,1,3,7) = \Pi(2,4,5,6)
      \end{equation}

    \subsection{b)}
      Indeksformer:
      \begin{equation}
        F(x,y,z) = \Sigma(0,3,7) = \Pi(1,2,4,5,6)
      \end{equation}
      Algebraisk sum av standardproduktform:
      \begin{equation}
        F(x,y,z) = \bar{x}\bar{y}\bar{z} + \bar{x}yz + xyz
      \end{equation}
      Algebraisk produkt av standardsumform:
      \begin{equation}
        F(x,y,z) = (x+y+\bar{z})(x+\bar{y}+z)(\bar{x}+y+z)(\bar{x}+y+\bar{z})(\bar{x}+\bar{y}+z)
      \end{equation}

    \subsection{c)}
      Indeksformer:
      \begin{equation}
        F(p,q,r)=\Pi(1,3,5,7)=\Sigma(0,2,4,6)
      \end{equation}
      Algebraisk sum av standardproduktform:
      \begin{equation}
        F(p,q,r)=\bar{p}\bar{q}\bar{r} + \bar{p}q\bar{r} + p\bar{q}\bar{r} + pq\bar{r}
      \end{equation}
      Algebraisk produkt av standardsumform:
      \begin{equation}
        F(p,q,r)=(p+q+\bar{r})(p+\bar{q}+\bar{r})(\bar{p}+q+\bar{r})(\bar{p}+\bar{q}+\bar{r})
      \end{equation}

  \section{Oppgåve 3}
    Setter uttrykket opp i funksjonstabell
    \begin{center}
      \begin{tabular}{ |c|c|c|c|c| }
        \hline
        Indeks & r & s & t & G(r,s,t) \\
        \hline
        0 & $0$ & $0$ & $0$ & 0 \\
        \hline
        1 & $0$ & $0$ & $1$ & 1 \\
        \hline
        2 & $0$ & $1$ & $0$ & 1 \\
        \hline
        3 & $0$ & $1$ & $1$ & 1 \\
        \hline
        4 & $1$ & $0$ & $0$ & 0 \\
        \hline
        5 & $1$ & $0$ & $1$ & 0 \\
        \hline
        6 & $1$ & $1$ & $0$ & 1 \\
        \hline
        7 & $1$ & $1$ & $1$ & 1 \\
        \hline
      \end{tabular}
    \end{center}
    $G(r,s,t)$ har mintermane 1,2,3,6,7. Desse kan vi sette inn i Karnaugh-diagrammet som
    einarar.
    \begin{center}
      \begin{karnaugh-map}[4][2][1][$st$][$r$]
        \minterms{1,2,3,6,7}
        \maxterms{0,4,5}
        %\indeterminants{}
        \implicant{1}{3}
        \implicant{3}{6}
        %\implicantcorner
        %\implicantedge{4}{12}{6}{14}
      \end{karnaugh-map}
    \end{center}
    ut ifrå dette diagrammet kan vi skrive om $G(r,s,t) = s + t\N{r}$
    \newpage
    Vi kan også fylle Karnaugh-diagrammet direkte.
    \begin{center}
      \begin{karnaugh-map}[4][2][1][$st$][$r$]
        \manualterms{0,$\bar{r}t$,$\scriptstyle{s\bar{t}+\bar{r}s}$,$\bar{r}s$,0,0,$s\bar{t}$,$rst$}
        \implicant{1}{3}
        \implicant{3}{6}
      \end{karnaugh-map}
    \end{center}

  \section{Oppgåve 4}
    \subsection{a)}
      Deler opp i to operasjonar for betre oversikt.
      \begin{center}
        \begin{karnaugh-map}[4][4][1][$cd$][$ab$]
          \minterms{0,1,2,4,5,6,8,9,10}
          \maxterms{3,7,11,12,13,14,15}
          \implicant{0}{5}
          \implicantcorner
        \end{karnaugh-map}
      \end{center}
      Raud: $\bar{a}\bar{c}$, grøn: $\bar{b}\bar{d}$
      \begin{center}
        \begin{karnaugh-map}[4][4][1][$cd$][$ab$]
          \minterms{0,1,2,4,5,6,8,9,10}
          \maxterms{3,7,11,12,13,14,15}
          \implicantedge{0}{4}{2}{6}
          \implicantedge{0}{1}{8}{9}
        \end{karnaugh-map}
      \end{center}
      Raud: $\bar{a}\bar{d}$, grøn: $\bar{b}\bar{c}$

      \begin{center}
        \begin{karnaugh-map}[4][4][1][$cd$][$ab$]
          \minterms{0,1,2,4,5,6,8,9,10}
          \maxterms{3,7,11,12,13,14,15}
          \implicant{0}{5}
          \implicantcorner
          \implicantedge{0}{4}{2}{6}
          \implicantedge{0}{1}{8}{9}
        \end{karnaugh-map}
      \end{center}
      $F(a,b,c,d) = \bar{a}\bar{c}+\bar{b}\bar{d}+\bar{a}\bar{d}+\bar{b}\bar{c}$

    \subsection{b)}
      Vi kan bruke Karnaugh-diagram til å finne produkt av sum-forma.
      \begin{center}
        \begin{karnaugh-map}[4][4][1][$cd$][$ab$]
          \minterms{0,1,2,4,5,6,8,9,10}
          \maxterms{3,7,11,12,13,14,15}
          \implicant{12}{14}
          \implicant{3}{11}
          %\implicant{3}{7}
          %\implicant{15}{11}
          %\implicant{13}{15}
          %\implicantedge{12}{12}{14}{14}
        \end{karnaugh-map}
      \end{center}
      vi finner at $F(a,b,c,d) = (\bar{a} + \bar{b})(\bar{c} + \bar{d})$
  \section{Oppgåve 5}
    \subsection{a)}
      Setter opp funksjonstabell for dekodaren med hardkoda verdiar for dei 
      ti siffersymbola.
      \begin{center}
        \begin{tabular}{ |c|c|c|c|c| c|c|c|c|c|c|c| }
          \hline
          Indeks & $B_3$ & $B_2$ & $B_1$ & $B_0$ & g & f & e & d & c & b & a \\
          \hline
          0 & $0$ & $0$ & $0$ & $0$ &    0 & 1 & 1 & 1 & 1 & 1 & 1 \\
          \hline
          1 & $0$ & $0$ & $0$ & $1$ &    0 & 0 & 0 & 0 & 1 & 1 & 0 \\
          \hline
          2 & $0$ & $0$ & $1$ & $0$ &    1 & 0 & 1 & 1 & 0 & 1 & 1 \\
          \hline
          3 & $0$ & $0$ & $1$ & $1$ &    1 & 0 & 0 & 1 & 1 & 1 & 1 \\
          \hline
          4 & $0$ & $1$ & $0$ & $0$ &    1 & 1 & 0 & 0 & 1 & 1 & 0 \\
          \hline
          5 & $0$ & $1$ & $0$ & $1$ &    1 & 1 & 0 & 1 & 1 & 0 & 1 \\
          \hline
          6 & $0$ & $1$ & $1$ & $0$ &    1 & 1 & 1 & 1 & 1 & 0 & 1 \\
          \hline
          7 & $0$ & $1$ & $1$ & $1$ &    0 & 0 & 0 & 0 & 1 & 1 & 1 \\
          \hline
          8 & $1$ & $0$ & $0$ & $0$ &    1 & 1 & 1 & 1 & 1 & 1 & 1 \\
          \hline
          9 & $1$ & $0$ & $0$ & $1$ &    1 & 1 & 0 & 0 & 1 & 1 & 1 \\
          \hline
          10 & $1$ & $0$ & $1$ & $0$ & - & - & - & - & - & - & - \\
          \hline
          11 & $1$ & $0$ & $1$ & $1$ & - & - & - & - & - & - & - \\
          \hline
          12 & $1$ & $1$ & $0$ & $0$ & - & - & - & - & - & - & - \\
          \hline
          13 & $1$ & $1$ & $0$ & $1$ & - & - & - & - & - & - & - \\
          \hline
          14 & $1$ & $1$ & $1$ & $0$ & - & - & - & - & - & - & - \\
          \hline
          15 & $1$ & $1$ & $1$ & $1$ & - & - & - & - & - & - & - \\
          \hline
        \end{tabular}
      \end{center}
      Forenkler dei sju funksjonane vha. Karnaugh-diagram med sikte på forma
      sum av standardprodukt
      \begin{equation}
        g(B_3,B_2,B_1,B_0) = \Pi(0,1,7)
      \end{equation}
      \begin{center}
        \begin{karnaugh-map}[4][4][1][$B_3B_2$][$B_1B_0$]
          \minterms{2,3,4,5,6,8,9}
          \maxterms{0,1,7}
          \implicant{12}{10}
          \implicant{4}{13}
          \implicant{2}{10}
          \implicantedge{3}{2}{11}{10}
          \indeterminants{10,11,12,13,14,15}
        \end{karnaugh-map}
      \end{center}
      \begin{equation}
        g(B_3,B_2,B_1,B_0) = B_1 + B_0\bar{B_3} + \bar{B_0}B_3 + B_3\bar{B_2}
      \end{equation}



      \begin{equation}
        f(B_3,B_2,B_1,B_0) = \Pi(1,2,3,7)
      \end{equation}
      \begin{center}
        \begin{karnaugh-map}[4][4][1][$B_3B_2$][$B_1B_0$]
          \minterms{0,4,5,6,8,9}
          \maxterms{1,2,3,7}
          \implicant{12}{10}
          \implicant{0}{8}
          \implicant{4}{13}
          \implicantedge{4}{12}{6}{14}
          \indeterminants{10,11,12,13,14,15}
        \end{karnaugh-map}
      \end{center}
      \begin{equation}
        f(B_3,B_2,B_1,B_0) = B_1 + \bar{B_3}\bar{B_2} + B_0\bar{B_3} + B_0\bar{B_2}
      \end{equation}


      \begin{equation}
        e(B_3,B_2,B_1,B_0) = \Sigma(0,2,6,8)
      \end{equation}
      \begin{center}
        \begin{karnaugh-map}[4][4][1][$B_3B_2$][$B_1B_0$]
          \minterms{0,2,6,8}
          \maxterms{1,3,4,5,7,9}
          \indeterminants{10,11,12,13,14,15}
          \implicant{2}{10}
          \implicantcorner
        \end{karnaugh-map}
      \end{center}
      \begin{equation}
        e(B_3,B_2,B_1,B_0) = \bar{B_2}\bar{B_0} + B_3\bar{B_2}
      \end{equation}


      \begin{equation}
        d(B_3,B_2,B_1,B_0) = \Pi(1,4,7,9)
      \end{equation}
      \begin{center}
        \begin{karnaugh-map}[4][4][1][$B_3B_2$][$B_1B_0$]
          \minterms{0,2,3,5,6,8}
          \maxterms{1,4,7,9}
          \indeterminants{10,11,12,13,14,15}
          \implicantcorner
          \implicant{5}{13}
          \implicant{2}{10}
          \implicantedge{3}{2}{11}{10}
        \end{karnaugh-map}
      \end{center}
      \begin{equation}
        d(B_3,B_2,B_1,B_0) = \bar{B_2}\bar{B_0} + B_3\bar{B_2} + B_3\bar{B_0} + B_0\bar{B_3}B_2
      \end{equation}


      \begin{equation}
        c(B_3,B_2,B_1,B_0) = \Pi(2)
      \end{equation}
      \begin{center}
        \begin{karnaugh-map}[4][4][1][$B_3B_2$][$B_1B_0$]
          \minterms{0,1,3,4,5,6,7,8,9}
          \maxterms{2}
          \indeterminants{10,11,12,13,14,15}
          \implicant{0}{9}
          \implicant{1}{11}
          \implicant{4}{14}
        \end{karnaugh-map}
      \end{center}
      \begin{equation}
        c(B_3,B_2,B_1,B_0) = \bar{B_3} + B_2 + B_0
      \end{equation}


      \begin{equation}
        b(B_3,B_2,B_1,B_0) = \Pi(5,6)
      \end{equation}
      \begin{center}
        \begin{karnaugh-map}[4][4][1][$B_3B_2$][$B_1B_0$]
          \minterms{0,1,2,3,4,7,8,9}
          \maxterms{5,6}
          \indeterminants{10,11,12,13,14,15}
          \implicant{0}{8}
          \implicant{3}{11}
          \implicantedge{0}{2}{8}{10}
        \end{karnaugh-map}
      \end{center}
      \begin{equation}
        b(B_3,B_2,B_1,B_0) = \bar{B_0} + \bar{B_3}\bar{B_2} + B_3B_2
      \end{equation}


      \begin{equation}
        a(B_3,B_2,B_1,B_0) = \Pi(1,4)
      \end{equation}
      \begin{center}
        \begin{karnaugh-map}[4][4][1][$B_3B_2$][$B_1B_0$]
          \minterms{0,2,3,5,6,7,8,9}
          \maxterms{1,4}
          \indeterminants{10,11,12,13,14,15}
          \implicant{3}{10}
          \implicant{12}{10}
          \implicant{5}{15}
          \implicantcorner
        \end{karnaugh-map}
      \end{center}
      \begin{equation}
        a(B_3,B_2,B_1,B_0) = B_1 + B_0B_2 + B_3 + \bar{B_0}\bar{B_2}
      \end{equation}
      Tilsaman har vi funksjonane:
      \begin{itemize}
        \item $a(B_3,B_2,B_1,B_0) = B_1 + B_0B_2 + B_3 + \bar{B_0}\bar{B_2}$
        \item $b(B_3,B_2,B_1,B_0) = \bar{B_0} + \bar{B_3}\bar{B_2} + B_3B_2$
        \item $c(B_3,B_2,B_1,B_0) = \bar{B_3} + B_2 + B_0$
        \item $d(B_3,B_2,B_1,B_0) = \bar{B_2}\bar{B_0} + B_3\bar{B_2} +
          B_3\bar{B_0} + B_0\bar{B_3}B_2$
        \item $e(B_3,B_2,B_1,B_0) = \bar{B_2}\bar{B_0} + B_3\bar{B_2}$
        \item $f(B_3,B_2,B_1,B_0) = B_1 + \bar{B_3}\bar{B_2} + B_0\bar{B_3} + B_0\bar{B_2}$
        \item $g(B_3,B_2,B_1,B_0) = B_1 + B_0\bar{B_3} + \bar{B_0}B_3 + B_3\bar{B_2}$
      \end{itemize}

    \subsection{b)}
      1111 vil vere 15 i BCD som er ein ugyldig verdi, men vi kan sjå kva som vert påtrykt
      i utgangane.
      \begin{itemize}
        \item $a(1,1,1,1) = 1+1+1+0=1$
        \item $b(1,1,1,1) = 0+0+1=1$
        \item $c(1,1,1,1) = 0+1+1=1$
        \item $d(1,1,1,1) = 0+0+0+0=0$
        \item $e(1,1,1,1) = 0+0=0$
        \item $f(1,1,1,1) = 1+0+0+0=1$
        \item $g(1,1,1,1) = 1+0+0+0=1$
      \end{itemize}
      a,b,c,f og g vil vere på, og displayet vil vise eit nital.

  \section{Ekstraoppgåve}
    $F(A,B,C,D)=\Pi(0,1,5,7,8,9,10,12,13,14,15)$
    \subsection{a)}
      Finner sum av produkt-uttrykket først
      \begin{center}
        \begin{karnaugh-map}[4][4][1][$CD$][$AB$]
          \minterms{2,3,4,6,11}
          \maxterms{0,1,5,7,8,9,10,12,13,14,15}
          \implicantedge{4}{4}{6}{6}
          \implicantedge{3}{3}{11}{11}
          \implicant{3}{2}
        \end{karnaugh-map}
      \end{center}
      \begin{equation}
        F(A,B,C,D)=\bar{A}B\bar{D} + \bar{B}CD + \bar{A}\bar{B}C
      \end{equation}
      setter opp nytt Karnaugh-diagram for å finne produkt av sum-uttrykket

      \begin{center}
        \begin{karnaugh-map}[4][4][1][$CD$][$AB$]
          \minterms{2,3,4,6,11}
          \maxterms{0,1,5,7,8,9,10,12,13,14,15}
          \implicant{5}{15}
          \implicantedge{12}{8}{14}{10}
          \implicantedge{0}{1}{8}{9}
        \end{karnaugh-map}
      \end{center}
      \begin{equation}
        F(A,B,C,D)=(\bar{B} + \bar{D})(\bar{A} + D)(B + C)
      \end{equation}
      
    \subsection{b)}
      Bruker P4a for å løyse opp parantesane
      \begin{equation}
        (\bar{B} + \bar{D})(\bar{A} + D)(B + C) =
        (\bar{B}B + \bar{B}C + \bar{D}B + \bar{D}C)(\bar{A}+D)
      \end{equation}
      bruker P5b og T2b for å fjerne det første leddet
      \begin{equation}
        (\bar{B}C + \bar{D}B + \bar{D}C)(\bar{A}+D)
      \end{equation}
      bruker P4a for å løyse opp parantesen
      \begin{equation}
        \bar{A}B\bar{D} + \bar{B}CD + \bar{A}\bar{B}C + \bar{D}DB + \bar{D}DC + \bar{A}C\bar{D}
      \end{equation}
      bruker P5b og T2b for å fjerne ledd fire og fem
      \begin{equation}
        \bar{A}B\bar{D} + \bar{B}CD + \bar{A}\bar{B}C + \bar{A}C\bar{D}
      \end{equation}
      Dette er det originale uttrykket men med eit ekstra ledd $\bar{A}C\bar{D}$. Om vi
      ser på Karnaugh-diagrammet eller funksjonstabell ser vi at dette leddet er redundant,
      og det kan derfor fjernast uten å påvirke funksjonen. Men korleis ein skal kunne fjerne
      dette leddet med reknereglar frå boolsk algebra veit eg ikkje.


\end{document}
