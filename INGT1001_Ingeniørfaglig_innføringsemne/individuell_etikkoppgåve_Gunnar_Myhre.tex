
\documentclass[12pt,a4paper]{article}
\title{%
	Trinn 1 - individuell tekst \\
	\large INGT1001 - Etikk og Metode
	}
\author{Gunnar Myhre, BIELEKTRO}

\usepackage{graphicx}
\usepackage[utf8]{inputenc}
\usepackage[norsk]{babel}
\usepackage{pgfplots}
\graphicspath{ {./images} }

\setlength{\parindent}{4em}
\setlength{\parskip}{1em}

\begin{document}
  \maketitle
I Noreg som i mange andre land har vi laga visse lovar og føringer som arbeids- og næringslivet må ordne seg under. Det finnast ein brei kategori av aktivitetar som myndigigheitene definerer som \textbf{arbeidslivskriminalitet}. Nokre av dei er utført med overlegg og kan vere grundig organisert, medan andre kan skyldast meir systemiske og underliggande mekanikkar i arbeidslivet og samfunnet som heilhet. Arbeids- og Sosialdepartementet uttaler at arbeidslivskriminalitet \textit{«utnytter arbeidstakarar eller virker konkurransevridande og undergrev samfunnsstrukturen»}\cite{dep}, og søker å bekjempe den med heimel i m.a. \textbf{Arbeidsmiljøloven}. I denne teksten ønsker eg å smått utforske temaet om arbeidslivskriminalitet, og nærare bestemt underkategorien \textbf{korrupsjon}, i eit juridisk og etisk perspektiv.

Korrupsjon går ut på å betale eller ta imot betaling for å utilbørlig oppnå makt, posisjonar og andre fordelar. Slike betalingar vert omtalt som bestikkingar, og kva som er \textit{utilbørlig} er opp til retten å tolke i det enkelte tilfelle \cite{snl}. Korrupsjon er nevnt fleire plassar i lovverket, m.a. i Straffelovens § 387-388 og i Arbeidsmiljølovens § 2 A-1 der det festast rett til å varsle om kritikkverdige forhold i ein virksomhet. Årleg faller det mange dommar med heimel i desse lovane, og sakane er ofte kompliserte forhold der økonomisk utruskap, heleri, underslag eller andre former for økonomisk kriminalitet er involvert. Til tross for at Noreg framstår som eit gjennomsiktig land med lite korrupsjon viser Transparency Internationals korrupsjonsindeks frå 2016 at Noreg kjem værst ut av landa i Skandinavia med ein poengsum på 85/100, bak Sverige (88/100) og Danmark (90/100). Indeksen for 2020 har liknande resultat med Noreg (84/100), Sverige (85/100) og Danmark (88/100).\cite{boka} \cite{transp}

Om vi skal diskutere korrupsjon ut ifrå eit \textbf{etisk} perspektiv kan det avdekke nokre predisposisjonar vi har til forholdet mellom \textbf{penger} og \textbf{makt}. Det er ingenting sjølvsagt i bruken av penger -- kontanter eller virtuelle – som byttemiddel i eit samfunn. Det er ein teknologi eller ei norm på linje med andre teknikkar og normer vi har applisert, og dermed åpen for granskning. Vanlegvis, og kanskje i sjølve \textit{essensen} av uttrykket kjem det at det skal vere ein korrelasjon mellom penger og \textbf{verdi}. Dette er også den vanlege oppfatning: Ein jobbar eller gjer seg på andre måtar fortent til å motta penger, og deretter er ein fri til å nytte desse pengene for å kjøpe seg goder eller tenester. Men nokre gonger er det ikkje samsvar mellom pengenes normerte verdi og den reelle verdien av goden eller tenesten den skal representere (om vi for ein augeblink kan tillate oss å snakke om \textit{reell verdi} utan å måtte fundamentere etikken ontologisk). Som ekstremt eksempel på dette kan vi trekke fram at nokre aktivitetar slik som tjuveri og annan vinningskriminalitet kan argumenterast for å ha \textit{netto negativ verdi} for samfunnet (sidan ellers unødvendige forsvarsverk må opprettast i eit tillitslaust samfunn osv.). Dette er også grunnen til at dagens moderne pengesystem er iscenesatt i samanheng med eit regulerande apparat i form av myndigheter og lovverk. Likevel er det problematisk å bestemme kva som skal lovfestast som økonomisk kriminalitet og ikkje. Straffelovens to korrupsjonsparagrafar (§ 387-8) legger mykje av den \textbf{normative} tyngden i sine formuleringer over på ordet utilbørlig, slik som i 
	\textit{«Med bot eller fengsel inntil 3 år straffes den som \textbf{a)} for seg eller andre krever, mottar eller aksepterer et tilbud om en utilbørlig fordel i anledning av utøvelsen av stilling, verv eller utføringen av oppdrag, eller \textbf{b)} gir eller tilbyr noen en utilbørlig fordel i anledning av utøvelsen av stilling, verv eller utføringen av oppdrag.»} \cite{straffelov}
Dette plasserer mykje av byrden for den normative tolkninga på rettsvesenet sidan kva som reknast som utilbørleg vil avgjere kva som  juridisk sett er korrupsjon.

Ein kan diskutere korrupsjon innanfor forskjellige moralske perspektiv. Om ein argumenterer med \textbf{moralsk non-kognitivisme} vil vi kun konstantere at sidan korrupsjon fordrer å akseptere eller gjeve ein utilbørleg fordel vil alle påstandar om at ein aktivitet er korrupsjon vere normative og dermed ikkje ha nokon sanningsverdi. Sjølv om dette utsegnet vil ha få ontologiske forpliktelsar og dermed vere vanskeleg å felle er det ikkje særleg konstruktivt i drøftinga av eit samfunnsproblem som korrupsjon. Vi kan argumentere \textbf{pliktetisk} med at \textit{personar med makt plikter å ikkje misbruke den for personleg vinnst}, eller \textbf{dydsetisk} med at \textit{ein god borgar skal ikkje misbruke si makt}. Begge desse utsegna kan vi vidare argumentere for med \textbf{moralsk kognitivisme} (det er fornuftig å ha satte reglar eller dydar for å handle etisk et.c.). Vi kan også med \textbf{konsekvensetisk} argumentasjon legge fram scenario med forskjellige gradar av frislipp eller innstrammingar i lovverket rundt økonomi og spesifikt korrupsjon, og sjå heilhetleg på kva for scenario vi ønskar mest. Samfunnet er veldig komplekst og det kan derfor vere best å applisere alle moglege etiske perspektiv for å tilnærme seg ein betre forståing av problemstillinga. Ein må også ikkje gløyme å sjå på motsetninga: Kva kan vere positivt med korrupsjon, og kva kan vere negativt med føretak imot korrupsjon (slik som strengare regulering og lovar).




\begin{thebibliography}{1}
  \bibitem{dep} https://www.regjeringen.no/no/dokumenter/strategi-mot-arbeidslivskriminalitet-2021/id2831867/
  \bibitem{snl} https://snl.no/korrupsjon
  \bibitem{boka} Ronny Kjelsberg: Teknologi og Vitenskap s. 314
  \bibitem{transp} https://www.transparency.org/en/cpi/2020/index/nzl
  \bibitem{straffelov} https://lovdata.no/dokument/NL/lov/2005-05-20-28/
  \bibitem{arblov} https://lovdata.no/dokument/NL/lov/2005-06-17-62/
  \bibitem{ssb1} https://www.ssb.no/sosiale-forhold-og-kriminalitet/faktaside/kriminalitet
  \bibitem{ssb2} https://www.ssb.no/sosiale-forhold-og-kriminalitet/artikler-og-publikasjoner/faerre-virksomheter-utsatt
\end{thebibliography}


	
\end{document}
